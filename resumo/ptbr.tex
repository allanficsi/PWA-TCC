% resumo em português
\setlength{\absparsep}{18pt} % ajusta o espaçamento dos parágrafos do resumo
\begin{resumo}
 
O desenvolvimento ágil está cada dia mais presente no cotidiano das empresas desenvolvedoras de software. A crescente busca por agilidade no desenvolvimento e a competitividade do mercado impactam na existência deste cenário. Portanto, muitas empresas buscam aplicar as metodologias definidas neste ramo de desenvolvimento em seu processos, porém, essa não é uma tarefa simples. A definição e implantação dessas práticas realizadas de maneira não suficientemente adequada podem trazer resultados adversos ao esperado. Portanto este trabalho teve como objetivo implantar a utilização de uma ferramenta de integração contínua em uma fábrica de software, o Núcleo de Práticas em Informática (NPI) da UFC - Campus Quixadá. Uma ferramenta de integração contínua está inserida em uma das práticas definidas pelo Extreme Programming (XP). Para a realização deste objetivo fora analisado o processo vigente executado no NPI, além da experiência do autor que era um membro desta fábrica de software. Somado a esta etapa foi definida a ferramenta de integração contínua a ser implantada de acordo com um conjunto de característica especificadas, o conhecimento dos membros da fábrica acerca de Gerência de configuração e Integração Contínua foi avaliado, e por fim, a implantação, com a definição de pontos positivos e negativos encontrados na implantação.


 \textbf{Palavras-chaves}: Integração Contínua. Desenvolvimento Ágil. Gerenciamento de Configuração.
\end{resumo}