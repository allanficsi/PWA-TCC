\chapter{Análise comparativa entre PWAs e Aplicações Móveis Nativas}

As \ac{PWA}s provaram ser muito úteis, e foram apresentados diversos recursos disponíveis em dispositivos móveis, em que as \ac{PWA}s dão suporte. No entanto, eles não estão aqui para tomar o lugar dos aplicativos nativos, mas para corrigir alguns problemas, como a compatibilidade entre plataformas e funcionamento em baixa conectividade e até mesmo offline.

Tendo isso em vista, esse capítulo irá abordar, o que são as aplicações para dispositivos móveis nativas, e uma comparação levando em consideração as vantagens e desvantagens entre desenvolver uma aplicação \ac{PWA} e um aplicativo nativo.

\section{Aplicação Nativa}
Os aplicativos nativos são desenvolvidos para um dispositivo específico ou uma plataforma, como Android e iOS, e podem interagir e aproveitar os recursos fornecidos por esse dispositivo específico. Esses aplicativos oferecem acesso mais rápido a diferentes recursos do dispositivo, como câmera, microfone, agenda, localização e etc.
Para usa-lós é necessário realizar a instalação do aplicativo no dispositivo desejado, normalmente os aplicativos se encontram nas lojas das plataformas, mas também é possível instalar via pacotes externos \cite{native}.

Ao desenvolver uma aplicação nativa, é necessário muitas vezes usar linguagens de programação específicas para a plataforma, elas permitem ao desenvolvedor ter acesso às funcionalidades de \ac{API} e hardware do dispositivo. As linguagens de programação disponíveis para desenvolver aplicativos nativos, varia para cada plataforma \cite{native}. Abaixo alguns exemplos das linguagens de programação disponíveis:

\begin{itemize}
	\item Java - Linguagem oficial do sistema operacional Android, usado para criar aplicativos nativos para a plataforma.
	\item Kotlin - Linguagem mais recente e semelhante ao Java, também para criar aplicativos nativos para Android.
	\item Objective-C - Linguagem principal para criar \textit{software} para dispositivos iOS.
	\item Swift -  Linguagem lançada da Apple para a criação de \textit{software} para o iOS, propõe-se a ser mais simples de implementar do que o Objective-C.
\end{itemize}

\section{Comparativo PWA e Aplicação nativa}
\subsection{Criação da aplicação e Distribuição}
\subsection*{Aplicação Nativa}
O desenvolvimento de uma aplicação móvel nativa para Android e iOS requer duas equipes, uma para cada sistema. Mesmo que os aplicativos de ambos os sistemas sejam desenvolvidos ao mesmo tempo, ainda demorará mais para garantir que a funcionalidade seja a mesma para os dois aplicativos. Tudo isso significa tempo e custos consideráveis necessários para criar um aplicativo.

O envio e aprovação via App Stores é uma parte separada da distribuição do aplicativo móvel nativo. O produto terá que passar por um período de moderação, pela equipe da plataforma, o que geralmente leva algum tempo. Para a Google Play Store, pode levar algumas horas, enquanto na Apple App Store pode levar de dois a quatro dias. Esse tempo de moderação, acaba impactando no atraso da distribuição da aplicação.

\subsection*{PWA}
A criação de uma \ac{PWA} requer apenas uma equipe de desenvolvimento da Web, pois na verdade é um site, embora com alguns recursos nativos dos dispositivos móveis. A validação pelas lojas não é necessária, pois se está criando um \textit{website}. Não é necessário enviar a aplicação para nenhuma loja nem esperar que ele seja aprovado. Depois que a \ac{PWA} é construída e publicada na Web, ela está pronto para ser usada.

\subsection{Instalação}
\subsection*{Aplicação Nativa}
Basicamente o fluxo de instalação é, ir em uma loja de aplicativos, procurar a aplicação e realizar o download, e após isso, a instalação no dispositivo móvel, e por fim o aplicativo estará disponível para uso.

\subsection*{PWA}
Com a \ac{PWA} é necessário abrir um navegador, acessar o site da aplicação, irá ser apresentado uma mensagem, perguntando se quer adicionar a aplicação para a tela inicial do dispositivo, confirmando, a aplicação irá ser instalada.

\subsection{Engajamento do Usuário}
Uma das ferramentas de engajamento mais poderosas é a \textit{Push Notifications}. São mensagens entregues por meio de um aplicativo instalado nos dispositivos, nos dispositivos móveis ou nos desktops dos usuários, para alertar seus usuários sobre novas chegadas de ações, vendas ou outras notícias.

\subsection*{Aplicação Nativa}
Em aplicativos móveis nativos, a disponibilidade do recurso de notificações por \textit{push} não depende do sistema operacional ou do modelo do dispositivo. Os usuários os receberão independentemente desses fatores.

\subsection*{PWA}
Nas \ac{PWA}, as \textit{Push Notifications} também estão disponíveis, mas apenas para o Android. Isso é possível graças aos \textit{service workers} eles podem enviar notificações quando uma aplicação \ac{PWA} não está em execução.

\subsection{Funcionamento Offline}
\subsection*{Aplicação Nativa}
Quando estamos falando sobre o modo offline do aplicativo nativo, assumimos que ele opera da mesma maneira que na conexão. O ponto é que um aplicativo nativo mostra o conteúdo e a funcionalidade que ele conseguiu armazenar em \textit{cache} quando a conexão ainda estava lá. Isso está disponível devido ao armazenamento local e à sincronização suave de dados com a nuvem.

\subsection*{PWA}
Nas aplicações \ac{PWA}, os usuários também podem aproveitar o modo offline. Quando ativadas, as páginas mostram o conteúdo pré-carregado ou carregado, que é fornecido com os \textit{service workers}. No entanto, o modo offline nas \ac{PWA}s é um pouco mais lento em comparação a um aplicativo móvel nativo, pois ele é implementado de forma diferente. Ao mesmo tempo, a diferença entre os dois tipos de aplicativos não é tão drástica.