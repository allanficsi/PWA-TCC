% ----------------------------------------------------------------------- %
% Arquivo: cap3.tex
% ----------------------------------------------------------------------- %
\chapter{Arquitetura dos Dispositivos Móveis}
\label{c_cap3}

Atualmente, os dispositivos móveis, são imprescindíveis na vida das pessoas. Quem imaginaria que no começo eles que tinham recursos muito limitados, poucas funcionalidades, se restringindo basicamente a serem calculadoras, e hoje, temos praticamente computadores pessoais em nossas mãos, tanto, em relação a funcionalidades quanto na questão de poder de processamento.

A medida que, os dispositivos móveis foram evoluindo, novas funcionalidades foram aparecendo, calendário, rádio, jogos, mas foi quando a Apple lançou seu primeiro smartphone, o Iphone, com seu próprio \ac{SO} o IOS, e a Google lançou o seu \ac{SO} para smartphones, o Android, que os dispositivos móveis ganhariam a fama de hoje. Com esses sistemas abriu se um leque de oportunidades para empresas e programadores avulsos, à desenvolverem os mais diversos aplicativos.

Com isso em mente, esse capítulo abordará um breve histórico da evolução dos dispositivos móveis, apresentará os \ac{SO}s mais famosos, mostrará como é desenvolver uma aplicação para cada \ac{SO} e analisará as diferenças entre elas.

\section{Evolução dos Dispositivos Móveis}


\section{Android}
\section{iOS}
\section{Comparação entre IOS e Android}




