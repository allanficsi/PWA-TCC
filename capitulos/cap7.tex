\chapter{\textbf{Conclusão}}

Analisando os resultados apresentados, as \ac{PWA}s, não chegaram para substituir as aplicações nativas, essa não é a sua proposta, mas sim facilitar o desenvolvimento de aplicações web, com recursos nativos de um dispositivo móvel, em que normalmente só era possível com aplicações nativas, e vão além disso, implementando recursos como o funcionamento offline e tratamentos customizados. O uso ou não da \ac{PWA} vai depender exclusivamente do negócio em que a aplicação vai ser direcionada, caso ela vá possuir diversas funcionalidades nativas, o ideal é desenvolver uma aplicação sem o uso da \ac{PWA}, entretanto, se uma \ac{PWA} possuir tais recursos, eu recomendo o uso da \ac{PWA}, por facilitar o desenvolvimento.

Por ser algo muito recente a maioria dos livros que abordam sobre as \ac{PWA}
começaram a serem publicados em meados de 2016, oferecendo na maioria das vezes uma visão
mas geral do assunto e com pouco foco na parte prática, com isso se fez necessário
procurar conteúdo nas mais variáveis fontes, como sites, artigos, e no próprio \textit{GitHub} . Com o intuito
de definir quais seriam os objetivos desse trabalho. Com a pesquisa bibliográfica foi
possível entender melhor a dimensão que as \ac{PWA} podem ter no futuro.

É importante salientar que entre os objetivos definidos nesse trabalho, a configuração do ambiente de
desenvolvimento foi o mais complexo. E a causa disso, foi porque a maioria do material disponível na Internet estava incompleta, e não mostrava como configurar por completo o
ambiente, outra ocorrência comum era o aparecimento de erros no ambiente durante
a execução de algum comando nos \textit{service workers}.


Sem dúvida as \ac{PWA}s, irão melhorar o engajamento do usuário, por exemplo, quando se utiliza um app mobile nativo, tem todo um passo à passo para começar a usá-lo. Primeiro tem que ir na loja de aplicativos do dispositivo móvel, fazer o download do app esperar a conclusão da instalação e após isso poder usá-lo, já com uma \ac{PWA} o usuário basta abrir um navegador, de sua preferência, e acessar a url do app, e pronto, já está disponível para utilização. Além do que, não é necessário 
esperar dias para que o novo app esteja disponível na loja de app, por exemplo, na Google Play.


