% ----------------------------------------------------------------------- %
% Arquivo: cap1.tex
% ----------------------------------------------------------------------- %

\chapter{Introdução}

O uso de dispositivos móveis ocupa grande parte de nossas vidas. Verificar os \textit{smartphones} várias vezes por dia tornou-se uma rotina para a maioria de nós. Durante anos, a única maneira das empresas atingirem os usuários de dispositivos móveis era criando um aplicativo móvel nativo ou híbrido. Hoje, porém, com a bordagem de \ac{PWA}s tornou-se uma solução alternativa para empresas de qualquer tamanho para atrair usuários móveis ativos.

\section{Justificativa}
Diante deste cenário, é necessário avaliar se uma aplicação para dispositivos móveis, irá fazer uso de um aplicativo nativo, ou se uma abordagem com \ac{PWA} é o suficiente para atender os requisitos.

Este trabalho, portanto, dá-se no sentido de apresentar as \ac{PWA}s, apresentando suas vantagens e desvantagens com relação a aplicativos nativos, com o objetivo de  servir como base para saber qual abordagem usar ao desenvolver uma aplicação.

\section{Organização Do Texto}
Esse trabalho está inicialmente organizado em 7 partes contando com essa seção,
o capítulo 2 apresenta como funciona a Arquitetura WEB, neste tópico será falado sobre a evolução da \textit{internet}, os modelos de multicamadas e os principais estilos arquitetônicos. No capítulo 3, é abordado a  evolução dos dispositivos móveis, e seus principais sistemas operacionais disponíveis no mercado.  A quarta parte desse trabalho, foca sobre as \ac{PWA}s, suas características, vantagens e desvantagens. No capítulo 5, é feito um estudo comparativo entre as aplicações móveis nativas e as {PWA}s. A penúltima parte desse trabalho foca na  configuração de um ambiente  e desenvolvimento de uma aplicação \ac{PWA}s.
E  o capítulo 7 é destinado às conclusões que foram conseguidas durante o desenvolvimento desse trabalho.

