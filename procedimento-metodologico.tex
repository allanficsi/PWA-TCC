\chapter{Procedimentos Metodológicos}\label{metodologia}

\section{Analisar as atividades do Núcleo de Práticas}
Esta atividade visa identificar como as atividades ocorrem dentro do Núcleo de Práticas em Informática, com o objetivo de gerar informações que servirão de complemento para a escolha da ferramenta a ser implantada de acordo com o contexto que será detectado nesta ação. Para tal foi analisado o processo\footnote{www.npi.quixada.ufc.br/processo/} existente modelado pela ferramenta \textit{EPF Composer}. Além da experiência do autor, pois este era um estagiário da organização com experiência de 8 meses nas atividades lá realizadas. 

\section{Pesquisar e selecionar a ferramenta de Integração Contínua}

Esta atividade consiste em  colher informações e selecionar a ferramenta que melhor se adapta a realidade existente no Núcleo de Práticas em Informática. 
Para a escolha da ferramenta foi preciso definir um conjunto de requisitos que a ferramenta deveria possuir e suprir, de modo a filtrar a ferramenta escolhida dentre as diversas existentes. As definições foram descritas \autoref{escolhaFerramenta}. 

\section{Absorção do perfil dos estagiários do Núcleo de Práticas}
Esta atividade tem como objetivo entender e identificar os conhecimentos dos estagiários do núcleo acerca de integração contínua, experiência de uso em projetos pessoais, conhecimento na ferramenta e como esta funciona. Para assim elaborar soluções a serem utilizadas na implantação da ferramenta como: acompanhamento, treinamento, etc. Para tal fora realizado um questionário online de escopo fechado que tinha como objetivo extrair o conhecimentos dos estagiários sobre integração contínua, seu uso e gerência de configuração de software.

\section{Implantação da ferramenta de integração contínua}
A implantação será realizada no núcleo de práticas e a utilização da ferramenta será aplicada a um projeto piloto que estará em desenvolvimento no momento da aplicação, de modo a avaliar o impacto que a ferramenta causará. Ao inicio será realizado um treinamento para explanação do funcionamento da ferramenta, bem como a utilização desta impactará nas atividades dos estagiários para assim, ser implantada.

\section{Coleta de Métricas do Código}
Esta etapa consiste na coleta contínua de métricas através da inspeção contínua de código do software em desenvolvimento utilizando a integração contínua. Para isso fora utilizado uma ferramenta de análise estática de código, o Sonarqube \footnote{www.sonarqube.org/} integrado ao Jenkins em um processo \textit{post-build}, onde o código será verificado e analisado com intuito de encontrar não conformidade que servirá de insumo para melhorias no código. Aumentando assim, sua qualidade, controle de acompanhamento do projeto e diminuição de esforços de retrabalho e futuras manutenções . 

\section{Análise dos dados coletados}
Esta atividade tem como objetivo analisar os dados das violações fornecidos pela ferramenta SonarQube, de modo a representar graficamente os dados quantitativos coletados do código.