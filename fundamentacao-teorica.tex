\chapter{Fundamentação Teórica}\label{fundamentacao}
Nesta seção será apresentado os conceitos necessários para um completo entendimento deste trabalho. A \autoref{gerenciadeconfiguracao} aborda os conceitos inerentes à gerência de configuração, juntamente com algumas ferramentas CASE (Computer-Aided Software Engineering), com ênfase em Integração Contínua descrito na \autoref{integracaocont} e por fim na \autoref{processonpi} será descrito o processo do NPI e uma ênfase no processo de Gerência de Configuração.

\section{Gerência de Configuração}\label{gerenciadeconfiguracao}
\section{Gerência de Configuração}\label{gerenciadeconfiguracao}
A gerência de configuração é a área da engenharia de software responsável pela evolução do software. Ela atua durante todo o ciclo de vida do produto de software e, por meio de técnicas, ferramentas e metodologias, visa garantir que as mudanças que irão ocorrer dentro do ciclo de vida do desenvolvimento do software sejam identificadas, avaliadas e comunicada a todos os envolvidos através de ferramentas que auxiliam neste processo de evolução.
Portanto "o propósito do processo de Gerência de Configuração é estabelecer e manter a integridade de todos os produtos de trabalho de um processo ou projeto e disponibilizá-la a todos os envolvidos"\space\cite{mpsbr}.
\subsection{Plano de Gerenciamento de Configuração}\label{pgc}
O Plano de Gerenciamento de Configuração (PGC) descreve todas as atividades de configuração e mudança que serão realizadas durante o projeto. Um conjunto de atividades, responsabilidades, ferramentas, recursos e etc. A gerência de configuração tem como objetivo garantir a integridade dos itens de configuração, que são quaisquer artefatos que estejam sob custódia da gerência de configuração, através do versionamento, da identificação e controle de mudanças e acesso \cite{pressman2010}. 

\subsection{Sistema de Controle de Versão}
Um sistema de controle de versão: "	[...] combina procedimentos e ferramentas para gerenciar diferentes versões de objetos de configurações que são criadas durante o processo de engenharia de software" \cite[p.~927]{pressman2010}.
Atualmente, o uso de sistemas de controle de versão se tornou comum nas empresas de grande e pequeno porte. Tais ferramentas permitem que se tenha o controle de diferentes versões de arquivos que estão submetidos ao versionamento, recuperação de versões antigas, visualização de alterações realizadas em arquivos e saber por quem, e quando o arquivo foi alterado. Através de comandos (i.e.,\textit{check-in},\textit{check-out}) os usuários conseguem se comunicar com o repositório a fim de obter os artefatos ali armazenados \cite{gleiph2011}.

Em situações especiais, faz-se necessário que os desenvolvedores trabalhem em uma linha diferente da original chamada de \textit{mainline} geralmente essa situação ocorre quando tem-se como objetivo a correção de \textit{bugs} de versões anteriores do repositório, nesse caso, um \textit{branch}, uma ramificação na linha de desenvolvimento do controle de versão, é criado afim de permitir a realização desta ação, concedendo assim o trabalho em paralelo sobre o mesmo repositório.
\begin{figure}[h]
\centering
\caption[Branch no Sistema de Controle de Versão]{Branch no Sistema de Controle de Versão.}
\includegraphics[width=0.5\linewidth]{./images/branch}
\label{fig:Branch}
\legend {\fontsize{10}{12}\selectfont {Fonte: \citeonline{tableless2012}}.}
\end{figure}
A \autoref{fig:Branch} demonstra a criação de um \textit{branch} paralelo à linha de desenvolvimento principal chamada de branch feature1 e branch master respectivamente. Posteriormente as ações realizadas no \textit{branch feature1} são incorporadas ao \textit{branch master}.

Os sistemas de controle de versão podem possuir três características: Local, Centralizado e Distribuído. Um sistema de controle de versão local armazena todas as informações de um arquivo submetido ao versionamento na máquina, localmente, guardando diferentes versões daquele arquivo. Enquanto um sistema de controle de versão centralizado como o nome diz, possui um único servidor centralizado, como o \textit{subversion} \footnote{http://subversion.apache.org}, \textit{perforce} \footnote{http://www.perforce.com} este tipo de padrão de SCV mantém em seu único servidor todos os arquivos versionados. Para cada comando de comunicação realizado nos arquivos versionados, uma requisição deverá ser feita, podendo gerar lentidão ou deixar o servidor fora de funcionamento. E por fim os sistemas de controle de versão distribuídos que possuem um servidor central onde os arquivos são submetidos ao versionamento, entretanto, cada desenvolvedor possui em sua máquina de trabalho as versões que estavam no servidor, tornando cada \textit{workstation} um "servidor", portanto, caso ocorra um problema no servidor central, estes podem ser recuperados via \textit{workstation}, mantendo assim, a integridade dos arquivos e evitando ser um ponto único de falha \cite{git}	.


\subsection{Sistema de Controle de Mudança}
Todo software sofre mudanças enfrentar essas mudanças é o papel da gerência de configuração, e para isso o gerente de configuração utiliza um sistema de controle de mudança. "O controle de mudança combina procedimentos humanos e ferramentas automatizadas para proporcionar um mecanismo de controle de mudança"\space \citeonline[p~.930]{pressman2010}. As mudanças devem ser avaliadas com cautela baseando-se, em seu custo benefício. Uma combinação de esforço e \textit{business value}. A mudança tem início quando um "cliente" solicita a mudanças através de um formulário, conhecido como \textit{change request}.

 Nesse formulário está descrito os aspectos da mudança. Após a solicitação ser realizada, esta deve ser avaliada, verificando se a mesma já foi solicitada, ou corrigida em caso de \textit{bugs}. Após a mudança ser validada, uma equipe de desenvolvedores avaliam os impactos que esta mudança tem sobre o sistema, verificando custo/benefício e esforço de realização \cite{sommerville2011}. Posterior a esta análise, a mudança será avaliada por um comitê de controle de mudança (CCB), que avaliará o impacto da perspectiva do negócio, que decidirá se esta mudança será revisada, aprovada ou reprovada. Alguns sistemas que fornecem este controle sobre as mudança são: \textit{Redmine \footnote{http://www.redmine.org}, GitHub \footnote{http://www.github.com} Jira \footnote{https://www.atlassian.com/software/jira}}
\subsection{Auditoria de Configuração}
"Uma auditoria de configuração de software complementa a revisão técnica formal ao avaliar um objeto de configuração quanto às características que geralmente não são consideradas durante a revisão"\space\citeonline[p~.934]{pressman2010}. Ela tem como objetivo garantir que mesmo com as mudanças realizadas no software, a qualidade foi mantida. As auditorias se dividem em dois tipos: auditorias funcionais e auditorias físicas, a auditoria física baseia-se em verificar se os itens de configuração estão devidamente atualizados e se as práticas e padrões foram realizados da maneira correta, enquanto a auditoria funcional busca verificar os aspectos lógicos dos itens de configuração.
\subsection{Ferramentas de Build}
As ferramentas de \textit{build} têm como objetivo automatizar processos repetitivos, aumentando a produtividade e facilitando o trabalho do desenvolvedor. Através da definição de uma rotina, ou conjunto de comandos, o desenvolvedor informa a ferramenta que tipo de processo ele deseja automatizar, podendo ser uma compilação, teste de classe, recriação de uma tabela nova no banco de dados, comprimir arquivos css e javascript. Cabe ao desenvolvedor definir o escopo da automatização. Alguns exemplo deste tipo de ferramenta são: \textit{Ant, Grunt, Gulp, Maven}.


\begin{figure}[H]
\centering
\caption[Processo Lógico de uma Build]{Processo Lógico de uma Build.}
\includegraphics[scale=0.7]{./images/build}
\label{fig:build}
\legend {\fontsize{10}{12}\selectfont {Fonte: \citeonline{paul2007}}.}
\end{figure}
Na figura \autoref{fig:build} um \textit{script} foi definido para realizar as seguintes funções: \textit{clean} no projeto, compilação do código fonte, integração com o banco de dados, execução dos testes, inspeções no código e por fim,o \textit{deploy} da aplicação.


%\subsection{Ferramentas de Integração Contínua}

\section{Integração Contínua}\label{integracaocont}
\begin{OnehalfSpace}
A integração contínua tem como objetivo identificar erros mais rapidamente, permitindo que alterações efetuadas e integradas aos repositórios dos sistemas de controle de versão sejam posteriormente verificadas e, caso erros ocorram, estes sejam notificados imediatamente ao autor da alteração.
A melhor definição acerca de integração contínua foi definida por \citeonline{fowler2000}
\end{OnehalfSpace}

\begin{citacao}
"[...] uma prática de desenvolvimento de software onde os membros de um time integram seu trabalho frequentemente, geralmente cada pessoa integra pelo menos diariamente – podendo haver múltiplas integrações por dia. Cada integração é verificada por uma \textit{build} automatizada (incluindo testes) para detectar erros de integração o mais rápido possível. Muitos times acham que essa abordagem leva a uma significante redução nos problemas de integração e permite que um time desenvolva software coeso mais rapidamente." \citeonline[tradução nossa]{fowler2000}.
\end{citacao}

\subsection{Características de Integração}
Os requisitos para utilização de uma ferramenta de integração contínua de acordo com \citeonline{Anti2010} são:
\begin{itemize}
\item {\textbf{Uma conexão com um sistema de controle de versão:}}

A integração contínua necessita desta conexão, pois ela identifica as alterações ocorridas no repositório, e inicia o processo de integração.

\item {\textbf{A definição de uma \textit{build}:}}

A integração contínua possui uma \textit{build} privada que será executada assim que o processo de integração for iniciado, e é esta \textit{build} que definirá quais ações serão realizadas no processo de integração, tais como compilação, testes, análise de código.
\item {\textbf{Um mecanismo de \textit{feedback}:}}

Um dos principais objetivos da integração contínua consiste em seu \textit{feedback} imediato, sendo assim, um mecanismo deste tipo é essencial para a ferramenta, tais como e-mail, sms.
\item {\textbf{Um processo de integração do código: }}
O processo de integração consiste em como este será realizado, se manualmente ou através de um servidor de integração contínua.

\end{itemize}

\subsection{Processo de Integração}
A integração ocorre quando alguma mudança é enviada ao sistema de controle de versão do repositório, que, através de um servidor de integração contínua identifica as mudanças e executa sua \textit{build} privada \cite{mraz2013}. 


\begin{figure}[H]
\centering
\caption[Ambiente de Integração Contínua]{Ambiente de Integração Contínua.}
\includegraphics[scale=1.0]{./images/CI}
\label{fig:CI}
\legend {\fontsize{10}{12}\selectfont Fonte: \citeonline{paul2005}.}
\end{figure}

A \autoref{fig:CI} descreve um ambiente em que um servidor de integração contínua é utilizado. Existem três ambientes de trabalho distintos formado por três desenvolvedores, que obtiveram uma cópia do projeto do repositório do SCV para trabalharem em suas \textit{workstation}. Durante o trabalho, alterações foram efetuadas e foi realizado o \textit{commit} ao repositório central. Após a inserção junto ao repositório, o servidor de integração contínua verifica as alterações e executa uma \textit{build} de integração. Caso exista um problema com a \textit{build} e esta não seja bem sucedida, o responsável pela alteração será informado sobre o ocorrido, assim seu objetivo em diante será a correção da \textit{build}.

\subsection{Benefícios da Integração Contínua}

Esta subseção tem como objetivo destacar os principais objetivos existentes na utilização de uma ferramenta de integração contínua.

As principais vantagens em utilizar um servidor de integração contínua segundo \citeonline[p.~29]{paul2007} são:

\begin{itemize}
\item {\textbf{Redução de Riscos}}: 
Através da detecção imediada de código quebrados, ou incorretos, reduz-se riscos atrelados ao produto.
\item {\textbf{Redução de processos manuais repetitivos}}:
Um conjunto de tarefas são executadas automaticamente pela build privada do servidor de integração contínua.
\item {\textbf{Permitir melhor visibilidade do projeto}}:
Através da identificação de informações inerentes ao desenvolvimento, como frequência de códigos defeituosos, módulos mais complexos, permitindo maior gerenciamento do projeto.
\item {\textbf{Estabelecer uma maior confiança no produto do time de desenvolvimento}}:
Através da visualizações de mudanças bem sucedidas, os desenvolvedores sentem maior confiança ao realizarem mudanças.

\item\textbf{Coleta de métricas a cada \textit{build}}: Um dos benefícios que podem ser obtidos através da ferramenta, é a coleta das métricas do código a ser desenvolvido e alterado. Através da utilização de uma ferramenta de análise estática de código é possível realizar uma análise de violações, não conformidades. E por meio desses indicadores, a equipe consegue ter um maior controle no desenvolvimento do software, investindo na melhoria contínua do software.
\end{itemize}	

%\subsection{Integração Contínua e a Redução de Riscos}
%Os Riscos em produtos de software estão diretamente relacionados. Segundo \citeonline[p~.48]{paul2007} se você consegue reduzir certos riscos no software, você pode melhorar a qualidade do software.
%\subsection{Builds Automatizadas}
%Builds são rotinas de execução definidas com o objetivo de reduzir processos repetitivos. Durante o processo de desenvolvimento de um software muitas ações tendem a serem repetidas por parte dos desenvolvedores, utilizar o tempo para a realização  de atividades que poderiam ser automatizadas, de forma manual, reduz a produtividade e preocupações com melhorias devido ao tempo "apertado". Somando-se a isso, uma build garante que tudo que está nela definido será executado, evitando assim, que determinada ação seja esquecida, ou caso um novo membro entre na equipe uma explicação do que ele deve fazer, ou não esquecer de fazer, não faz-se necessário.

\subsection{Integração Contínua Manual}
Na integração contínua manual o processo de integração é realizado individualmente, possibilitando que apenas um desenvolvedor realize \textit{check-in} no repositório durante o intervalo de integração \citeonline{gleiph2011}. Este tipo de abordagem permite que apenas uma pessoa realize o \textit{check-in}, assim, as integrações serão contínuas e seguidas, não paralelas. Este tipo de abordagem garante uma maior confiabilidade nas integrações, pois segue um padrão de integração e os itens do repositório possuem maior consistência, garantindo que a estrutura do repositório seja mantida \cite{gleiph2011}.

\subsection{Integração Contínua Automatizada}
A integração contínua automatizada é auxiliada pelo uso de um servidor de integração contínua, que obtém do controle de versão as alterações realizadas e executa sua \textit{build} privada com o objetivo de verificar possíveis erros gerados por essas modificações.
\begin{citacao}
Integração contínua automática possui a vantagem de ser escalável e, deste  modo, oferecer  maior  suporte  ao  trabalho  colaborativo.  Com  a utilização de Servidores de IC, a responsabilidade  de realizar construções da integração é retirada  dos desenvolvedores. Portanto, os desenvolvedores podem realizar \textit{check-in} sem a necessidade de conquistar a vez de integrar. Esse fator é fundamental para que os  \textit{check-ins}  continuem sendo verificados sem a necessidade de um desenvolvedor realizar a construção e identificar problemas, resultando na eliminação do gargalo humano. \citeonline[p~.54]{gleiph2011}. 
\end{citacao}

\subsection{Processo de Escolha da Ferramenta}\label{escolhaFerramenta}
Esta seção define um conjunto de características que auxiliam no processo de escolha de uma ferramenta de integração contínua.
\begin{itemize}
\item {\textbf{Suporte à Linguagem:}}

O processo de escolha de um servidor de integração continua deve ser baseado de acordo com o suporte a linguagem, visto que alguns sistemas são construídos para trabalharem com uma linguagem de programação específica.

\item {\textbf{Suporte ao Sistema de Controle de Versão:}}

Como explanado anteriormente, a importância do SCV dentro de um servidor de integração contínua é altíssima, portanto escolher uma ferramenta que integre-se com o repositório é essencial, pois alguns servidores fornecem suporte a SCV mais populares, como \textit{Subversion}, \textit{Git}, entretanto pode não haver suporte ao \textit{Mercurial} por exemplo.


\item {\textbf{Segurança:}}

Garantir que somente pessoas autorizadas devem ter acessos aos artefatos existentes no servidor de integração contínua.

\item {\textbf{Extensibilidade:}}

Capacidade da ferramenta ter funcionalidades adicionadas por meio de \textit{plugins}, ser extensível.

\item {\textbf{Usabilidade:}}

Possuir baixa dificuldade na realização de ações dentro da ferramenta, boa aprendizagem, compreensibilidade.

\item {\textbf{Instalação e Configuração:}}

Facilidade de instalação em diferentes ambientes de operação, tais como sistemas operacionais, hardware através da utilização de recursos. Documentação clara e objetiva do processo de instalação.


\end{itemize}


\section{Métricas de Software}

Uma métrica de software é uma característica de um determinado sistema de software, desenvolvimento, processo ou documentação, de modo que possa ser medido \cite{sommerville2011}.

As métricas de software são dados quantitativos que informaram o estado de um sistema de software. Através desses dados, é possível um maior controle e tomada de decisão pela parte gerencial de uma organização de software \cite{karina2008}. São exemplos de métricas segundo \citeonline{merson2014}:
\begin{itemize}
\item{Cobertura de Testes}
\item{Complexidade Ciclomática}
\item{Débito Técnico}
\end{itemize}	

Dentre o conjunto de métricas avaliados, as violações estão inseridas. Violações são não conformidades encontradas em um conjunto de regras definidas.

As violações podem ser classificadas em diferentes graus de severidade:

\begin{itemize}
\item{Critical:} Causam perda de dados, vulnerabilidades de segurança, tornam o sistema inutilizável.
\item{Major: Erros que impactam uma minoria de usuários do sistema.}
\item{Normal: Erros que afetam uma parte da funcionalidade do sistema.}
\item{Minor: Estão mais relacionados ao estilo do código.}
\end{itemize}	


\section{Processo do NPI}\label{processonpi}
O NPI possui um modelo de processo definido, este processo é baseado nos modelos e metodologias Scrum, MPS.BR e XP. Este modelo define as práticas e o modelo de trabalho dos envolvidos nas atividades do núcleo. Dentro do modelo de processo definido no NPI \footnote{http://www.npi.quixada.ufc.br/processo/} este trabalho tem como objetivo focar no modelo de processo de gerência de configuração.

O NPI subdivide-se em dois turnos, manhã e tarde, sendo cada turno supervisionado por um professor supervisor diferente. Estes turnos podem ou não estar trabalhando no mesmo projeto, embora o mais comum é que trabalhem em projetos diferentes. As equipes contam com em média oito membros onde comumente destes, dois são alocados para as atividades de requisitos e testes, um para liderança técnica, enquanto o restante da equipe é alocado para as atividades de desenvolvimento, incluindo o líder técnico. O professor supervisor tem como papel o auxílio aos líderes técnicos, acompanhamento do projeto, avaliação dos estagiários, escolha dos projetos a serem desenvolvidos pelas equipes e usualmente realizar o papel de \textit{Product Owner}. 

O líder técnico possui papel gerencial bem como de desenvolvimento, suas atribuições partem desde a condução de reuniões, resolução de conflitos,atribuição e definições de tarefas, até o acompanhamento das atividades. 

A \autoref{fig:processo-npi} mostra o processo utilizado no NPI modelado através da ferramenta EPF Composer. Na figura	existem duas atividades que ocorrem em paralelo, são elas: Avaliação do Processo e Iniciar Projeto, este que subdivide-se em mais três atividades, a primeira delas a atividade de Requisitos, que posteriormente fornece entrada para um ciclo de \textit{Sprints} que ocorrerá enquanto houver funcionalidades não implementadas, simultaneamente com a atividade de Requisitos estão de Gerenciamento do Projeto e o Gerenciamento de Configuração.
\begin{figure}[H]
\centering
\caption[Processo do NPI]{Processo do NPI.}
\includegraphics[scale=0.8]{./images/processo-npi}
\label{fig:processo-npi}
\legend {\fontsize{10}{12}\selectfont {Fonte: \citeonline{processonpi}}.}
\end{figure}


\subsection{Processo de Gerência de Configuração do NPI}
O modelo de processo\footnote{http://www.npi.quixada.ufc.br/processo/} relacionado a gerência de configuração é descrito na \autoref{fig:procnpi}. Este modelo de processo possui duas atividades que serão descritas abaixo:
\begin{itemize}
\item \textbf{Criar Plano de Gerenciamento de Configuração:} Esta atividade é realizada pelo líder técnico da equipe envolvida. Esta atividade subdivide-se em quatro etapas são elas:
\begin{itemize}
\item \textbf{Identificar Itens de Configuração:} Esta atividade caracteriza-se pela criação, especificação e seleção dos produtos de trabalho, ferramentas, itens que tem objetivo descrever os produtos de trabalho. Exemplos de itens desta atividade são: Requisitos, Diagramas, Testes.

\item \textbf{Atribuir Identificadores únicos para os itens de configuração:} Esta atividade possui um nome bem sugestivo tem como intuito atribuir a cada item de configuração um identificador único de modo a facilitar a identificação dentro do projeto. O identificador segue o padrão $\left[PROJETO\right]$-$\left[TIPO\right]$-EXTRA.EXTENSÃO. Como exemplo um artefato possuiria o seguinte identificador: $\left[GPA\right]$-$\left[REQ\right]$-Especificacao.doc

\item \textbf{Identificar o responsável por cada item de configuração:} Esta atividade tem como objetivo atribuir a cada item de configuração um responsável, permitindo assim, uma maior facilidade na identificação do responsável de um determinado item de configuração.
\item \textbf{Criar	Plano de Gerenciamento de Configuração:} Esta atividade tem como objetivo a elaboração do PGC explicado na \autoref{pgc} por meio dos dados obtidos com as tarefas anteriores. O plano define os responsáveis pelas atividades de Gerência de Configuração, ferramentas e ambientes a serem utilizados e todos os itens de configuração identificados.
\end{itemize}
\item \textbf{Estabelecer um sistema de Gestão de Configuração:} Esta atividade tem como requisito que o plano de gerenciamento de configuração esteja concluído, e possui apenas uma etapa:
\begin{itemize}
\item \textbf{Estabelecer um sistema de Gestão de Configuração:} Esta atividade tem como objetivo definir as ferramentas de acesso, ambiente de armazenamento e métodos para criação e alteração dos itens de configuração \cite{processonpi}. 
\end{itemize}
\end{itemize}

\begin{figure}[H]
\centering
\caption[Processo de Gerenciamento de Configuração]{Processo de Gerenciamento de Configuração.}
\includegraphics[scale=1.3]{./images/processonpi}
\label{fig:procnpi}
\legend {\fontsize{10}{12}\selectfont Fonte: \citeonline{processonpi}.}
\end{figure}

