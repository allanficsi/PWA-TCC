%%%%%%%%%%%%%% Como usar o pacote acronym
% \ac{acronimo} -- Na primeira vez que for citado o acronimo, o nome completo irá aparecer
%                  seguido do acronimo entre parênteses. Na proxima vez somente o acronimo
%                  irá aparecer. Se usou a opção footnote no pacote, entao o nome por extenso
%                  irá aparecer aparecer no rodapé
%
% \acf{acronimo} -- Para aparecer com nome completo + acronimo
% \acs{acronimo} -- Para aparecer somente o acronimo
% \acl{acronimo} -- Nome por extenso somente, sem o acronimo
% \acp{acronimo} -- igual o \ac mas deixando no plural com S (ingles)
% \acfp{acronimo}--
% \acsp{acronimo}--
% \aclp{acronimo}--

\chapter*{Lista de abreviaturas e siglas}%
% \addcontentsline{toc}{chapter}{Lista de abreviaturas e siglas}
\markboth{Lista de abreviaturas e siglas}{}

\begin{acronym}
	\acro{API}{Interface de Programação de Aplicações}
	\acro{CERN}{Organização Europeia para a Pesquisa Nuclear}
	\acro{CSS}{\textit{Cascading Style Sheets}}
	\acro{DOM}{Modelo de Objeto de Documento}
	\acro{HTML}{Linguagem de Marcação de Hipertexto}
	\acro{HTTP}{Protocolo de Transferência de Hipertexto}
	\acro{JS}{JavaScript}
	\acro{JSON}{Notação de Objetos JavaScript}
	\acro{PWA}{\textit{Progressive Web App}}
	\acro{SGBD}{Sistema Gerenciador de Banco de Dados}
	\acro{SO}{Sistema Operacional}
	\acro{UI}{\textit{User Interface}}
	\acro{W3C}{\textit{World Wide Web Consortium}}
	\acro{WWW}{\textit{World Wide Web}}
\end{acronym}
