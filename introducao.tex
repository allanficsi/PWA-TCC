\chapter{Introdução}
% ----------------------------------------------------------
\vspace{-2em}
Nas últimas décadas, as metodologias ágeis atraíram uma grande atenção por parte das empresas desenvolvedoras de software. Isto deve-se muito à relação das metodologias ágeis com o conjunto de boas práticas pertencentes a ela, que adicionam qualidade, controle e garantia ao produto.


A Integração Contínua é uma dessas práticas, definida como uma ferramenta de gestão de configuração, que auxilia os desenvolvedores e permite que as mudanças que foram realizadas no software sejam imediatamente avaliadas, testadas e verificadas, de modo a prover um \textit{feedback} imediato para correção de possíveis erros de integração. Integrações, estas, que seriam verificadas apenas futuramente, posterior a problemas mais complexos de integração. \cite{paul2007}.

A metodologia ágil Scrum é utilizada no Núcleo de Práticas em Informática (NPI). T	rata-se de um ambiente onde estudantes que estão em fase de conclusão de curso podem estagiar e aprimorar seus conhecimentos adquiridos no decorrer de sua vida acadêmica, além de concluírem seus componentes curriculares obrigatórios. Este surgiu, devido à pouca demanda de empresas de Tecnologia da Informação (TI) na região onde a universidade se encontra e a crescente exigência por profissionais dotados de experiência em desenvolvimento de software \cite{npi2013}.

Os projetos lá desenvolvidos têm como objetivo construir soluções que facilitem as atividades do cotidiano da universidade, esta que tem um grande interesse no desenvolvimento destes projetos, pois consegue reduzir custos ao priorizar construções de sistemas internamente \cite{npi2013}. Em paralelo, se pode obter um aumento da qualidade dos profissionais formados, além de proporcionar um ambiente real de trabalho que tem como intuito facilitar a entrada dos concludentes no mercado de trabalho.

No NPI existe um modelo de processo definido em que os desenvolvedores devem seguir para o exercício de suas atividades \cite{npi2013}. Entretanto, este em momentos não é devidamente seguido, e, em seu processo a utilização de uma ferramenta de integração contínua não está introduzida, podendo ocasionar uma despadronização na maneira como estes  desenvolvedores trabalham em seus projetos. Somado-se a isto, o NPI apresenta problemas tais como: "Baixa qualidade da documentação dos sistemas; [\ldots] Falta de uma equipe de manutenção; [\ldots] Rotatividade dos profissionais" \cite[p.~4]{paduelli2006}.

A experiência em aplicar ferramentas de gestão de configuração foi abordada por \citeonline{poliana2007}, eles inseriram a utilização destas com o intuito de evitar a inserção de novos erros oriundos de manutenções realizadas no software. Diferentemente do trabalho a ser desenvolvido nesse projeto que busca relatar a experiência obtida na implantação de uma ferramenta de integração contínua.

Este trabalho tem como objetivo implantar uma ferramenta de integração contínua no Núcleo de Práticas em Informática da Universidade Federal do Ceará do Campus de Quixadá e como objetivos específicos estudar e analisar ferramentas de integração contínua, e selecionar a que melhor se adapta ao Núcleo de Práticas em Informática;
selecionar e implantar uma ferramenta de integração contínua automatizada e por fim coletar relatos e resultados provenientes da implantação da ferramenta.
