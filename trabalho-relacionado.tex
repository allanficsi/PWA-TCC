\chapter{Trabalhos Relacionados}\label{trabalhorel}
Nesta seção será descrito trabalhos que influenciaram os conceitos envolvidos neste trabalho, além de demonstrar pontos comuns e distintos entre si e o proposto.

O trabalho de \citeonline{pereira2013} descreve a implantação de uma ferramenta de integração contínua em um departamento de desenvolvimento e pesquisa, o Sedna, de uma empresa de engenharia, em um ambiente de MPS.BR nível F. Onde os principais clientes do Sedna são voltados a área de óleo e gás.

Durante o período de implantação, o Sedna fornecia manutenção a três sistemas, onde um tratava-se de uma aplicação web, deste modo o \textit{deploy} da aplicação para todos os seus usuários era de responsabilidade do Sedna. Tal ação tornava-se bastante custosa devido ao grande número de web sites que deveriam ser atualizados.
	
Dentro do Sedna algumas ferramentas de gerência de configuração já eram utilizadas, tais como o Atlassian Jira, Subversion (SVN) e o Atlassian Confluence. Ainda com a utilização destas ferramentas a equipe possuía grandes dificuldades no tempo de realização do \textit{deploy}, pois esta atividade consumia uma grande parte do tempo da equipe, tempo de aprendizagem e realização do \textit{deploy} da aplicação, principalmente para novos membros da equipe, e um \textit{feedback} atrasado para problemas básicos de commits errôneos.

Os autores descrevem que os pontos positivos da integração estão a utilização de uma ferramenta de integração contínua da mesma empresa que fornecia o sistema de gerenciamento de projetos, a ferramenta utilizada foi o Atlassian Bamboo e a utilização de ferramentas de automação do processo de build, o que facilitava o trabalho da integração contínua. Bem como o autor destaca as experiências negativas da implantação, que estão na ausência de uma máquina com requisitos mínimos exigidos para a utilização, a ausência de treinamento da equipe, onde os conhecimentos de IC estavam com o líder de projetos e o gerente de configuração, e por fim a ferramenta não fornecia suporte ao \textit{redeploy} dos artefatos gerados, o que gerou uma barreira na equipe acerca da ferramenta.


Uma das principiais vantagens da utilização da integração contínua é o seu \textit{feedback} imediato acerca de problemas de integração, e, entender e interpretar as principais causas dos problemas de integração foi realizado por \citeonline{miller2008}. Quando este percebeu que em um projeto da Microsoft, o Service Factory, a maior causa de falha na build eram violações no sistema de análise de código seguido por testes automatizados e erros de compilação.

A obtenção de métricas através da coleta contínua de métricas é uma grande vantagem incluída dentro da integração contínua, pois a sua principal vantagem é o aumento da qualidade do produto ao facilitar o processo de manutenção do software. Para isso \citeonline{moreira2010} desenvolveram um \textit{framework} para a extração de métricas automatizadas, implementado em um ambiente de integração contínua, logo, evidenciou-se a importância das inspeções e teste de códigos nas \textit{builds}, como foi descrito anteriormente.

Os testes são fundamentais na qualidade do software, e na confiança do produto em uma \textit{build} de integração contínua. Assim \citeonline{kim2009} propuseram a criação de um \textit{framework} automatizado de testes que permitisse facilitar a  construção dos casos de testes, reuso de componentes, relatórios mais legíveis e integrado ao ambiente de integração contínua.

Implantar uma integração contínua em um ambiente de desenvolvimento ágil foi realizado por \citeonline{abdul2012}. Estes propuseram um conjunto de boas práticas e coletaram experiências acerca desta implantação. Estes afirmam que os engenheiros podem não ser tão facilmente convencidos de aceitar a integração contínua, independentemente se esta é implementada \textit{top-down}, quando parte de uma hierarquia mais elevada da organização para as camadas inferiores, ou \textit{bottom-up}, quando parte dos desenvolvedores até os níveis mais altos da hierarquia, e que em ambas as abordagens existem prós e contras.
