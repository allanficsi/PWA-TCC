\include{fixos/pacoteseclass}
\setlrmarginsandblock{3cm}{2cm}{*}
\setulmarginsandblock{3cm}{2cm}{*}
\checkandfixthelayout
\pagestyle{myheadings}

\usepackage{float}
%\usepackage[a4paper,bottom=2cm,top=3cm,left=3cm,right=2cm]{geometry}
\usepackage{acronym}

% Informações de dados para CAPA e FOLHA DE ROSTO
\titulo{\uppercase{PWA}}
\autor{Pedro Cruz}
\local{Fortaleza, Ceará}
\data{2019}
\orientador{Prof Msc. Fred Viana}


\instituicao{%
FACULDADE LOURENÇO FILHO

\par
BACHARELADO EM SISTEMAS  DE  INFORMAÇÃO 
}
%\tipotrabalho{Trabalho de Conclusão de Curso (Monografia)}
%\preambulo{Trabalho de Conclusão de Curso submetido à Coordenação do Curso de Engenharia de Software do Campus Quixadá da Universidade Federal do Ceará, como requisito parcial para obtenção do Título de Bacharel em Engenharia de Software.}


\begin{document}
\frenchspacing 

% ----------------------------------------------------------
% ELEMENTOS PRÉ-TEXTUAIS
% ----------------------------------------------------------
\pretextual
% Capa
\imprimircapa
% Folha de rosto (* indica que haverá a ficha bibliográfica)
\imprimirfolhaderosto*

% Ficha Bibliográfica
%\include{fixos/fichabibliografica}

% Errata
%\include{editaveis/errata}

% Folha de Aprovação
%\include{editaveis/folhadeaprovacao}
%\imprimirfolhadeaprovacao

% Dedicatória
%\include{editaveis/dedicatoria}

% Agradecimentos
%\include{editaveis/agradecimentos}

% Epígrafe
%\include{editaveis/epigrafe}

% RESUMOS
%\include{resumo/ptbr}
%\include{resumo/us}
%\include{resumo/fr}
%\include{resumo/es}

% Lista de ilustrações
\pdfbookmark[0]{\listfigurename}{lof}
\listoffigures*
\cleardoublepage


% Lista de tabelas
\pdfbookmark[0]{\listtablename}{lot}
\listoftables*
\cleardoublepage

% Lista de Siglas
\pdfbookmark[0]{Lista de abreviaturas e siglas}{loa}
%%%%%%%%%%%%%% Como usar o pacote acronym
% \ac{acronimo} -- Na primeira vez que for citado o acronimo, o nome completo irá aparecer
%                  seguido do acronimo entre parênteses. Na proxima vez somente o acronimo
%                  irá aparecer. Se usou a opção footnote no pacote, entao o nome por extenso
%                  irá aparecer aparecer no rodapé
%
% \acf{acronimo} -- Para aparecer com nome completo + acronimo
% \acs{acronimo} -- Para aparecer somente o acronimo
% \acl{acronimo} -- Nome por extenso somente, sem o acronimo
% \acp{acronimo} -- igual o \ac mas deixando no plural com S (ingles)
% \acfp{acronimo}--
% \acsp{acronimo}--
% \aclp{acronimo}--

\chapter*{Lista de abreviaturas e siglas}%
% \addcontentsline{toc}{chapter}{Lista de abreviaturas e siglas}
\markboth{Lista de abreviaturas e siglas}{}

\begin{acronym}
	\acro{API}{Interface de Programação de Aplicações}
	\acro{CERN}{Organização Europeia para a Pesquisa Nuclear}
	\acro{CSS}{\textit{Cascading Style Sheets}}
	\acro{DOM}{Modelo de Objeto de Documento}
	\acro{HTML}{Linguagem de Marcação de Hipertexto}
	\acro{HTTP}{Protocolo de Transferência de Hipertexto}
	\acro{JS}{JavaScript}
	\acro{JSON}{Notação de Objetos JavaScript}
	\acro{PWA}{\textit{Progressive Web App}}
	\acro{SGBD}{Sistema Gerenciador de Banco de Dados}
	\acro{SO}{Sistema Operacional}
	\acro{UI}{\textit{User Interface}}
	\acro{W3C}{\textit{World Wide Web Consortium}}
	\acro{WWW}{\textit{World Wide Web}}
\end{acronym}

\newpage

\renewcommand*{\cftsectionfont}{\normalfont}
\renewcommand*{\cftchapterpagefont}{\normalfont}
% Abreviaturas e Siglas
%\include{editaveis/siglas}

% Símbolos
%\include{editaveis/simbolos}

% Sumário
\pdfbookmark[0]{\contentsname}{toc}
\tableofcontents*
\cleardoublepage

% ----------------------------------------------------------
% ELEMENTOS TEXTUAIS
% ----------------------------------------------------------
\textual

% ----------------------------------------------------------
% Introdução (exemplo de capítulo sem numeração, mas presente no Sumário)
% ----------------------------------------------------------
% ----------------------------------------------------------------------- %
% Arquivo: cap1.tex
% ----------------------------------------------------------------------- %

\chapter{{\Large\textbf{Introdução}}}

O uso de dispositivos móveis ocupa grande parte da vida das pessoas. Verificar os smartphones várias vezes por dia tornou-se uma rotina para a maioria delas. Durante anos, a única maneira das empresas atingirem os usuários de dispositivos móveis era criando um aplicativo móvel nativo ou híbrido. Hoje, porém, com a bordagem de \ac{PWA} tornou-se uma solução alternativa para empresas de qualquer tamanho para atrair usuários móveis ativos.


Diante deste cenário, é necessário avaliar se uma aplicação para dispositivos móveis, irá fazer uso de um aplicativo nativo, ou se uma abordagem com \ac{PWA} é o suficiente para atender os requisitos.

Este trabalho tem como objetivo geral, abordar as \ac{PWA}, apresentando suas vantagens e desvantagens com relação a aplicativos nativos, com o intuito de  servir como base para saber qual abordagem usar ao desenvolver uma aplicação. Como objetivos específicos têm-se:  abordar conceitualmente e funcionalmente as \ac{PWA},
os impactos que elas podem causar no desenvolvimento como um todo, já que pode-se está presenciando o fim de aplicativos nativos, e por fim configurar um ambiente de desenvolvimento onde será criado uma \ac{PWA}, fazendo uso de tecnologias como IndexedDb, Firebase e JavaScript.

Esse trabalho está organizado em sete capítulos contando com esta introdução,
o capítulo 2 apresenta como funciona a Arquitetura WEB, neste tópico será falado sobre a evolução da \textit{internet}, os modelos de multicamadas e os principais estilos arquitetônicos. No capítulo 3, é abordado a  evolução dos dispositivos móveis, e seus principais sistemas operacionais disponíveis no mercado.  A quarta parte desse trabalho, foca sobre as \ac{PWA}, suas características, vantagens e desvantagens. No capítulo 5, é feito um estudo comparativo entre as aplicações móveis nativas e as {PWA}. A penúltima parte desse trabalho foca na  configuração de um ambiente  e desenvolvimento de uma aplicação \ac{PWA}.
E  o capítulo 7 é destinado às conclusões que foram conseguidas durante o desenvolvimento desse trabalho.

%introdução
% ----------------------------------------------------------------------- %
% Arquivo: cap2.tex
% ----------------------------------------------------------------------- %

\chapter{\textbf{Arquiteturas  WEB}}
\label{c_cap2}

Uma aplicação \textit{web} é um \textit{software} que roda no lado do cliente, ou seja, na máquina do próprio usuário, com essa abordagem é necessário  apenas que o cliente possua um \textit{browser}(navegador), independente de seu sistema operacional para acessar uma aplicação \textit{web}. Com isso os desenvolvedores não precisam desenvolver diversas versões de uma mesma aplicação, para sistemas operacionais diferentes.

Tendo isso em vista, este capítulo visa apresentar um breve histórico da evolução da arquitetura  \textit{web}, apresentará o modelo de multicamadas e por fim demonstrará seus estilos arquitetônicos.

\section{História}


A Internet mudou completamente quando Tim Berners-Lee, a fim de solucionar os problemas de comunicações da \ac{CERN}, propôs que se usasse um sistema de comunicação em hipertexto. Esse sistema acabou agradando os gerentes do \ac{CERN}, e no ano seguinte, ele foi implantado com o nome de \textit{\ac{WWW}}. Todos os serviços da \textit{Internet} se renderam ao poder da \textit{Web} e à linguagem \ac{HTML}, que a sustenta. Até o serviço de correio eletrônico, campeão de tráfego na \textit{Internet} por muitos anos, que por muito tempo exigia aplicações específicas, separadas do \textit{browser}, hoje é lido dentro de um \textit{browser}, através de páginas \ac{HTML} \cite{rocha99}. 

A medida que o \ac{HTML} foi ganhando fama, quem a usava não queria apenas uma simples apresentação de texto estruturada, queriam usar cores, imagens e técnicas de design mais avançado, suas páginas \ac{HTML} acabavam ficando com vários códigos de estilos. Só que a manutenção seria o maior problema dessa abordagem, se um componente mudar de estilo em apenas uma página a alteração seria simples, mas se fosse em várias páginas, a manutenção já seria bastante penosa. Misturar estilo e estrutura não era mais interessante, e foi assim que em 1995, Håkon Wium Lie e Bert Bos apresentaram a proposta do \ac{CSS} que logo foi apoiada pela \ac{W3C}. A ideia geral era, utilizar \ac{HTML} somente para estruturar o \textit{website} e a tarefa de apresentação fica com o \ac{CSS} disposto em um arquivo separado com o sufixo .css ou no próprio \ac{HTML} demarcado pelas tags <style> \cite{devmediaCSS2018}.

O ano de 1995 é muito importante para a \textit{internet}, a Netscape Communications apresenta o \textit{Javascript}, uma linguagem leve, interpretada e baseada em objetos com funções de primeira classe, mais conhecida como a linguagem de \textit{script} para páginas \textit{Web}, mas usada também em vários outros ambientes sem \textit{browser} como \textit{node.js},  \textit{Apache CouchDB} e\textit{ Adobe Acrobat}. O \textit{Javascript} é uma linguagem de script multi-paradigma,  baseada em protótipo que é dinâmica, e suporta estilos de programação orientado a objetos, imperativo e funcional \cite{mozilla2018}, e possibilitou que os desenvolvedores pudessem criar componentes dinâmicos, e assim possibilitando uma melhora na \textit{interface} do usuário. Com o \textit{javascript} a \textit{internet} acabou se tornando mais performática e produtiva, porque os dados não precisavam mais ser enviados ao servidor para se gerar toda a página \ac{HTML}. Juntos o \textit{Javascript}, \ac{HTML} e \ac{CSS} são às três tecnologias mais populares para a produção de conteúdo na internet \cite{devsaran2018}.

Já no ano de 1996 a Macromedia lançaria o \textit{Flash}, que é uma plataforma multimídia de desenvolvimento para aplicações que contenham animações, áudio e vídeo, bastante utilizada na construção de anúncios publicitários e páginas \textit{web} interativas. O \textit{Flash} pode manipular vetores e gráficos para criar textos animados, desenhos, imagens e até \textit{streaming} de áudio e vídeo pela internet. 
O \textit{Flash} ganhou bastante popularidade entre os programadores e desenvolvedores \textit{web} por permitir um desenvolvimento rápido de aplicações com alta qualidade e integração transparente com outras categorias de conteúdo \cite{canaltechFlash2018}.

 Com o \textit{Flash} os programadores puderam dar ao usuário, uma experiência mais rica através de animações de áudio e vídeo. Além disso, as interações feitas no \textit{Flash} eram todas no lado do cliente, não precisando se comunicar com o servidor. Era comum que os \textit{sites} antes dos anos 2000, usasse conteúdo multimídia incorporado, o que por muitas vezes gerava uma poluição visual muito grande, e como consequência disso a popularidade do \textit{Flash} acabou caindo, as páginas acabaram ganhando um visual padrão, o acesso do usuário não era mais interrompido por áudio, vídeos e anúncios inesperados, a medida que seu uso foi diminuindo a página ficava mais leve e o uso de rede também \cite{devmediaAsync2018}.


 Apesar de  seu uso ir diminuindo gradualmente, o \textit{Flash} por muitos anos foi usado para criação de jogos e aplicativos interativos para dispositivos móveis. No ano de 2008 o primeiro rascunho público do \ac{HTML}5 é lançado pelo \textit{WHATWG} \cite{fasthosts2018}, essa tecnologia ganhou os holofotes, foi quando em 2010 o CEO da Apple Inc., Steve Jobs emitiu uma carta pública intitulada "Reflexões sobre o Adobe Flash", onde ele conclui que o desenvolvimento do \ac{HTML}5 tornaria o \textit{Flash} não mais necessário para exibir qualquer conteúdo \textit{web}. Depois disso o fim do \textit{Flash} era questão de tempo, muitos \textit{browsers} já não o utilizavam mais, e em julho de 2017, em comunicado oficial a Adobe, anunciou que vai encerrar definitivamente o suporte para sua tecnologia Flash até 2020. A companhia disse precisar destes últimos anos para incentivar criadores de conteúdo e desenvolvedores a utilizarem novas plataformas, principalmente formatos de código aberto e que funcionem também em celulares e \textit{tablets}  \cite{canaltechFlashFim2018}.

Em 1999, o conceito de aplicação \textit{web} apareceu na linguagem Java. Mais tarde, em 2005, o Ajax foi apresentado por Jesse James Garrett em seu artigo "Ajax: uma nova abordagem para a aplicação Web". Esta nova técnica de desenvolvimento, possibilitou a criação de aplicações \textit{web} assíncronas, ou seja, o fluxo do código não é interrompido até que a resposta seja obtida. Ao invés disso, após realizar a requisição, a resposta é obtida em um momento posterior, de forma independente, e tratada por uma função (chamada função de \textit{callback}). Com isso era possível enviar dados para o servidor e recuperá-los sem interferir na navegação de uma página específica, não precisando baixar a página inteira \cite{devmediaAsync2018}.


\section{Modelo de Multicamadas}
\label{s_c2_figuras}


\subsection*{Modelo de uma camada}

Esse modelo foi fortemente usado durante os anos 1960 até meados dos anos 80, também chamado de sistemas centralizados ou de arquitetura uni processada, o modelo de uma camada era caracterizado por manter todos os recursos do sistema (banco de dados, regras de negócios e interfaces de usuário) em computadores de grande porte, os conhecidos mainframes. Os terminais clientes não possuíam recursos de armazenamento ou processamento, sendo conhecidos como terminais burros ou mudos. Nesta arquitetura, o mainframe tinha a responsabilidade de realizar todas as tarefas e processamento \cite{devmediaMultiCamada2018}. As aplicações eram escritas em um único módulo ou camada monolítica, com programas e dados firmemente entrelaçados. Tal integração estreita entre programa e dados dificultava a evolução e reuso de componentes. Cada desenvolvedor de aplicação escolhia como estruturar e armazenar os dados e usava-se frequentemente técnicas para minimizar o caro armazenamento e manipulação em memória principal \cite{devmediaMultiCamadaP12018}.


\subsection*{Modelo de duas camadas}

Também conhecido como modelo cliente-servidor, seu uso se deu bastante durantes os anos 1980 à 1990, durante essa época novas tecnologias foram desenvolvidas e adotadas como, \textit{softwares} para gerenciamento de banco de dados relacionais (SGBD), a utilização de rede locais interligando microcomputadores departamentais, o início do paradigma da programação orientada a objetos, e a redução dos custos do hardware, tornando possível a massificação da computação pessoal \cite{devmediaMultiCamadaP12018}. Também devido à grande expansão das redes de computadores, os métodos para desenvolvimento de \textit{software} foram aos poucos evoluindo para uma arquitetura descentralizada, na qual não somente o servidor era responsável pelo processamento, mas as estações clientes também assumem parte desta tarefa. Dentro deste contexto que surgiu o modelo de duas camadas, justamente com o objetivo de dividir a carga de processamento entre o servidor e as máquinas clientes \cite{devmediaMultiCamada2018}. A \autoref{f_c2_cliente_servidor}  é uma representação do modelo cliente-servidor.

\begin{figure}[h!]
\centering
\caption{Modelo Cliente-Servidor}
\frame{\includegraphics[scale=0.60]{images/cliente-servidor}}\\
Fonte: \cite{devmediaMultiCamadaP12018}
\label{f_c2_cliente_servidor}
\end{figure}

Na camada do cliente, é onde o usuário interage com o sistema, ela é responsável por prover uma \ac{UI} agradável e de fácil acesso para o usuário possa manipular o sistema. Mas apesar de prover a \ac{UI}, a camada do cliente não se restringe a isso, regras de negócio também podem ser implementadas, assim diminuindo a complexidade no servidor. A camada do cliente, de acordo com a sua  complexidade pode ser definida de duas formas segundo \cite{devmediaMultiCamadaP12018}. Sendo elas:

\begin{itemize}
    \item Cliente Gordo
    \begin{itemize}
        \item Maior complexidade de regras de negócio
        \item Menos processamento para o servidor
        \item Possivelmente mais tráfego na rede
        \item Cliente é mais sensível a mudanças
    \end{itemize}
    \item Cliente Magro
    \begin{itemize}
        \item Menor complexidade de regras de negócio
        \item Mais processamento para o servidor
        \item Menor tráfego na rede
        \item Manutenção mais simples
    \end{itemize}
\end{itemize}

Com o modelo de duas camadas foi possível que \textit{softwares} de terceiros tivessem acesso ao banco de dados, ou seja, com isso os \textit{softwares} poderiam ser diferentes para cada usuário ou setor, afim de melhor atender às suas necessidades. Outra vantagem é a questão do custo-beneficio, as estações dos clientes são mais baratas do que um servidor \cite{devmediaMultiCamadaP12018}.

O modelo de duas camadas também pode apresentar algumas desvantagens, como sua aplicação é dividida em partes, acaba acarretando em um \textit{software} mais complexo, com novos cenários a serem tratados. A comunicação do cliente com o servidor se dá por meio da rede, com isso dados sensíveis serão trafegados na rede e precisam de um cuidado maior com a criptografia \cite{rocha99}. 

\subsection*{Modelo de Multicamadas}
Modelo de multicamadas ou cliente servidor de múltiplas camadas, se consolida a partir dos anos 90 como resposta a  popularização da internet e melhoria das tecnologias de redes, sendo  uma evolução do modelo de duas camadas. Ele tem como propósito  que uma aplicação cliente não realizasse comunicação direta com o banco de dados, no meio do caminho haveria uma ou mais camadas, que elas sim se comunicariam com o banco de dados. A ideia básica é distribuir o processamento da aplicação em várias máquinas, evitando a sobrecarga sobre uma única camada, como ocorria no modelo cliente-servidor. Com a distribuição da carga de processamento em diversas máquinas é possível melhorar o desempenho e compartilhar recursos, utilizando-os como se fossem recursos locais, característica conhecida como  transparência de uso \cite{devmediaMultiCamadaP12018}. Com isso a aplicação pode ser dividida em pequenos pedaços, cada um com sua responsabilidade. Além dessas vantagens, há uma compensação no custo em relação ao desempenho, a possibilidade de aumento de escala e expansão da rede sem perda de qualidade, melhora na robustez em função da distribuição dos serviços em mais de uma máquina, e muitos outros benefícios \cite{devmediaMultiCamadaP12018}.

Em uma aplicação que faz uso do  modelo de multicamadas, faz se necessário ao menos três camadas: camada de apresentação, camada de regras de negócio e a camada de dados. A \autoref{f_c2_multicamada} apresenta o esquema de um sistema multicamada. 

\begin{figure}[!htpb]
	\centering
	\caption{Modelo Multicamada}
	\label{f_c2_multicamada}
	\includegraphics[width=14cm]{images/multicamada.jpg}\\
    Fonte: \cite{diegomacedoArqApp}
 
\end{figure}

\subsection{Camada de Apresentação}

A camada de apresentação fica fisicamente localizada na estação cliente e é responsável por fazer a interação do usuário com o sistema. É uma camada bastante leve, que basicamente executa os tratamentos de telas e campos e geralmente acessa somente a segunda camada, a qual faz as requisições ao banco de dados e devolve o resultado. É também conhecida como cliente, regras de interface de usuário ou camada de interface \cite{devmediaMultiCamada2018}.

\subsection{Camada de Regra de Negócio}

Também conhecida como servidor de aplicação, lógica de negócio ou camada de acesso a dados, essa camada é a responsável por intermediar a comunicação entre a camada de apresentação com a camada de dados, só ela tem acesso a camada de dados. O servidor de aplicação é, geralmente, uma máquina dedicada e com elevados recursos de hardware, uma vez que nele é que ficam armazenados os métodos remotos (regras de negócios) e é realizado todo o seu tratamento e processamento. \cite{devmediaMultiCamada2018}.

\subsection{Camada de Dados}

Também chamada de camada de banco de dados, essa camada é onde se localiza o \ac{SGBD}, ela é responsável por receber requisições da camada de regra de negócio, interpretá las e assim executá-las no banco de dados  \cite{devmediaMultiCamada2018}.


\section{Vantagens}

O modelo de multicamadas apresentou diversas vantagens em relação ao modelo de duas camadas, são essas as que mais se destacam:

\subsection{Clientes Leves}

Diferentemente do modelo de duas camadas, onde a regra de negócio era divida tanto na camada do cliente quanto na do servidor, aqui a camada intermediária é quem fica a cargo disso, na camada de apresentação será apenas para visualização de dados, além de possuir possíveis tratamentos de campos e telas  \cite{devmediaMultiCamada2018}.

\subsection{Facilidade de Redistribuição}

As estações clientes acessam a mesma camada intermediária, sendo assim, quando houver novas implementações ou alterações nas regras de negócios, será refletido para todas as estações clientes  \cite{devmediaMultiCamada2018}.

\subsection{Modularização}

A modularização refere-se a separar a lógica do negócio e regras de acesso ao banco de dados da camada de apresentação. Desta maneira, várias aplicações clientes podem compartilhar as mesmas regras, que ficam encapsuladas em uma camada de acesso comum.Assim sendo, as regras ficam centralizadas em um único local, ao contrário de em uma aplicação desenvolvida em duas camadas, na qual geralmente existe redundância nestas regras e uma mudança mesmo que pequena acarretará na redistribuição do aplicativo em cada estação cliente \cite{devmediaMultiCamada2018}.

\subsection{Economia de conexões no servidor}

Em um sistema com o modelo de duas camadas, as estações clientes se comunicavam diretamente ao servidor que se localizava o banco de dados, então para cada estação cliente conectada era uma conexão aberta com o banco de dados, sendo que o banco de dados possuí um limite para conexões, tendo isso em vista, no modelo de multicamadas isso não ocorre já que quem se conecta com o banco de dados é o servidor de aplicação, localizado em uma camada intermediária, e uma conexão realizada pelo servidor de aplicação é compartilhada para as estações clientes a ele conectado, sendo assim poderia se ter várias estações clientes requisitando recursos do banco de dados, mas apenas uma conexão estaria aberta já que quem ficaria responsável pela conexão é o servidor de aplicação \cite{devmediaMultiCamada2018}.

\subsection{Independência de localização}

A localização não é um empecilho para a comunicação entre camadas, a estação cliente pode estar fisicamente distante para acessar as camadas intermediárias  \cite{devmediaMultiCamada2018}.

\subsection{Escalabilidade}

No modelo de duas camadas, quando um grande número de estações clientes se conectam ao servidor, acaba ocorrendo uma grande perda de desempenho. Já com o modelo de multicamadas este problema pode ser contornado, já que é possível replicar a regra de negócio em servidores distintos através do balanceamento de carga, isso quer dizer que quando um servidor de aplicação estiver sobrecarregado, outro servidor é acionado para ajudar no controle de conexões, isso também pode ser usado quando um servidor de aplicação parar de funcionar \cite{devmediaMultiCamada2018}.

\section{Estilos Arquitetônicos}

Nesta seção será apresentados os estilos arquitetônicos de desenvolvimento de uma aplicação web.

\subsection{Arquitetura Monolítica}

Segundo \cite{monoVsMicro2017},  "Uma aplicação monolítica é aquele tipo de aplicação na qual toda a base de código está contida em um só lugar, ou seja, todas as funcionalidades estão definidas no mesmo bloco". Esse bloco geralmente se divide em três partes: apresentação, negócio e dados. A \autoref{f_c2_app_monolitica} apresenta o modelo dessa arquitetura.

\begin{figure}[h]
	\centering
	\caption{Aplicação Monolitica}
	\includegraphics[scale=0.7]{images/app-monolitica.png}\\
	Fonte:\cite{monoVsMicro2017}
 	\label{f_c2_app_monolitica}
\end{figure}

\newpage
\subsection*{Camadas}

\subsection*{Apresentação}
Camada responsável pela visualização ou interface, que será apresentada para o usuário. "Em uma aplicação web, esta camada contém as páginas \ac{HTML} com \ac{JS} e \ac{CSS} que serão renderizadas no \textit{browser} de quem as acessar",  \cite{monoVsMicro2017}.

\subsection*{Negócio}
Camada que é responsável pela lógica da aplicação. Segundo \cite{monoVsMicro2017} "Nesta camada geralmente se encontram todas as bases de código, chamadas, \ac{API}'s e literalmente toda a inteligência do sistema em questão".

\subsection*{Dados}
Camada em que se encontram as classes encarregadas pela conexão com o \ac{SGBD} ou outro tipo de sistema de armazenamento de dados \cite{monoVsMicro2017}.

\subsection*{Vantagens}
\begin{itemize}
	\item Fácil deploy da aplicação.
	\item Não há duplicidade de código.
	\item Aplicação é desenvolvida usando uma mesma tecnologia.
	\item Seu desenvolvimento tende a ser mais rápido, por ser uma arquitetura mais simples.
	\item Fluxo de deploy simples.
\end{itemize}

\subsection*{Desvantagens}
\begin{itemize}
	\item Ponto único de falha.
	\item Aumento de tamanho e complexidade ao longo do tempo.
	\item Falta de flexibilidade, já que usa uma mesma tecnologia.
	\item Baixa escalabilidade, tendo que copiar toda a aplicação para escalar horizontalmente.
	\item Alta dependência de componentes.
	\item Dificuldade de alterações em produção, qualquer mudança se faz necessário, a reinicialização de todo o sistema.
	\item Demora de aculturamento, um novo desenvolvedor pode ter dificuldades para entender o funcionamento de um componente.
\end{itemize}

\subsection{Arquitetura de Micro serviços}
Uma arquitetura de micro serviços segundo \cite{microservices2014}:  \begin{citacao}"é uma abordagem que desenvolve uma aplicação única como uma suíte de pequenos serviços, cada um rodando em seu próprio processo e se comunicando com mecanismos leves, geralmente uma \ac{API} de recurso \ac{HTTP}" \end{citacao} Desse jeito nós separamos uma aplicação monolítica em pequenas aplicações autônomas, ou seja, devem possuir um sistema de deploy automático e independente, além de que cada aplicação possui um conjunto de regras de negócio específico. A \autoref{f_c2_app_monolitica_vs_microservices} mostra a comparação entre a arquitetura monolítica e de micro serviços.



\begin{figure}[!htpb]
	\centering
	\caption{Representação de módulos monolítico e micro serviços}
	\includegraphics[width=15cm]{images/micro-deployment.png}\\
	Fonte: \cite{microservices2014}
 	\label{f_c2_app_monolitica_vs_microservices}
\end{figure}

\subsection*{Vantagens}
\begin{itemize}
	\item Arquitetura individual simples.
	\item Sistemas totalmente independentes.
	\item Ausência de ponto de falha única.
	\item Fácil deploy da aplicação e testes unitários.
	\item Módulos podem usar tecnologias distintas.
	\item Serviços coesos e desacoplados.
	\item Facilidade de alterações em ambiente de produção, apenas o serviço específico é necessário ser reinicializado para refletir as alterações.
	\item Escalabilidade do sistema, serviços que sofrem de alta demanda podem ser replicados individualmente.
\end{itemize}

\subsection*{Desvantagens}
\begin{itemize}
	\item Desempenho prejudicado pela latência da rede e pelo custo de serialização e deserialização.
	\item Se a aplicação não for bem documentada, a arquitetura geral tende a ser tornar complexa.
	\item Falta de planejamento e má execução, a arquitetura pode se tornar uma grande bagunça.
	\item Repetição de código nos serviços.
\end{itemize}

%arquiteturas web
% ----------------------------------------------------------------------- %
% Arquivo: cap3.tex
% ----------------------------------------------------------------------- %
\chapter{Arquitetura dos Dispositivos Móveis}
\label{c_cap3}

Atualmente, os dispositivos móveis, são imprescindíveis na vida das pessoas. Quem imaginaria que no começo eles que tinham recursos muito limitados, poucas funcionalidades, se restringindo basicamente a serem calculadoras, e hoje, temos praticamente computadores pessoais em nossas mãos, tanto, em relação a funcionalidades quanto na questão de poder de processamento.

A medida que, os dispositivos móveis foram evoluindo, novas funcionalidades foram aparecendo, calendário, rádio, jogos, mas foi quando a Apple lançou seu primeiro smartphone, o Iphone, com seu próprio \ac{SO} o IOS, e a Google lançou o seu \ac{SO} para smartphones, o Android, que os dispositivos móveis ganhariam a fama de hoje. Com esses sistemas abriu se um leque de oportunidades para empresas e programadores avulsos, à desenvolverem os mais diversos aplicativos.

Com isso em mente, esse capítulo abordará um breve histórico da evolução dos dispositivos móveis, apresentará os \ac{SO}s mais famosos, mostrará como é desenvolver uma aplicação para cada \ac{SO} e analisará as diferenças entre elas.

\section{Evolução dos Dispositivos Móveis}


\section{Android}
\section{iOS}
\section{Comparação entre IOS e Android}




%Arquitetura do MOBILE
\chapter{Conhecendo as Progressives Web Apps}

\ac{PWA} é uma tecnologia emergente do Google, um conceito relativamente novo no mundo dos dispositivos móveis e da internet. De acordo com o Google Developers \cite{pwa}: \begin{citacao}
"As Progressive Web Apps fornecem uma experiência instalável, semelhante a um aplicativo, em computadores e dispositivos móveis que são criados e entregues diretamente pela Web. Eles são aplicativos da web que são rápidos e confiáveis. E o mais importante, são aplicativos da web que funcionam em qualquer navegador."
\end{citacao}

 \ac{PWA}s são desenvolvidas usando certas tecnologias e abordagens para criar aplicações que aproveitam os recursos dos dispositivos móveis nativos e de aplicativos da Web, essencialmente é uma mistura de aplicações web e mobiles nativas.

O Google definiu um \textit{checklist} a ser seguido para se considerar uma aplicação como uma \ac{PWA} são elas:

\begin{itemize}
	\item Progressivo - Funciona para qualquer usuário, independentemente do navegador escolhido, pois é criado com aprimoramento progressivo como princípio fundamental.
	\item Responsivo - Se adéqua a qualquer formato: desktop, celular ou tablet.
	\item Independente de conectividade - Aprimorado com \textit{service workers} para trabalhar \textit{off-line} ou em redes de baixa qualidade.
	\item Semelhante a aplicativos - Parece com aplicativos para os usuários, com interações e navegação de estilo de aplicativos, pois é compilado no modelo de \textit{shell} de aplicativo.
	\item Atual - Sempre atualizado graças ao processo de atualização do \textit{service worker}.
	\item Seguro - Fornecido via HTTPS para evitar invasões e garantir que o conteúdo não seja adulterado.
	\item Descobrível - Pode ser identificado como “aplicativo” graças aos manifestos W3C e ao escopo de registro do \textit{service worker}, que permitem que os mecanismos de pesquisa os encontrem.
	\item Reenvolvente - Facilita o reengajamento com recursos como notificações \textit{push}.
	\item Instalável - Permite que os usuários “guardem” os aplicativos mais úteis em suas telas iniciais sem precisar acessar uma loja de aplicativos.
	\item Linkável - Compartilhe facilmente por URL, não requer instalação complexa.
\end{itemize}

\section{Características}
\label{s_c4_caractetisticas}

\subsection{Confiável}
Quando iniciado a partir da tela inicial do usuário, os \textit{service workers} permitem que uma \ac{PWA} seja carregado instantaneamente, independentemente do estado da rede \cite{pwa}.


Basicamente os \textit{service workers} são \textit{scripts} que o navegador roda por debaixo dos panos, separado de uma página Web. Possibilitando que recursos sejam acessados mesmo sem uma interação do usuário ou de uma página Web. O \textit{service worker} possui hoje funcionalidades como \textit{Push Notifications}, que é basicamente, uma notificação que o usuário recebe sem requisita-lá, e Sincronização em Segundo Plano, que é uma \ac{API} que permite que a aplicação adie ações até que o usuário tenha uma conectividade com a internet estável. Isso é útil para garantir que uma ação feita pela o usuário, seja realmente realizada \cite{servicework} ???????????????.
O \textit{service worker} é uma \ac{API} interessante para os desenvolvedores já que permite configurar experiências off-line. Os S\textit{ervice Workers} possuem algumas características importantes:

\begin{itemize}
	\item É executado em uma \textit{thread} separada do navegador, portanto, não possui acesso ao \ac{DOM} diretamente.
	\item É um \textit{proxy} de rede programável, portanto, permite gerenciar as solicitações de rede da página.
	\item É encerrado quando ficar ocioso e reiniciado quando for necessário.
	\item Os \textit{service workers} utilizam promessas para retornar os dados de suas funções.
\end{itemize}

O \textit{Service Worker} possui um ciclo de vida à parte da página da Web e funciona em uma estrutura já determinada. Entendendo como cada evento ocorre e suas respostas é o ideal para se ter a melhor forma de atualizar e guardar os arquivos \cite{pwa2} ????. Primeiramente é necessário registrar o service worker na aplicação, isso pode ser feito via um arquivo \ac{JS}, pode se adicionar uma verificação, antes do registro, para verificar se o navegador suporta o uso de \textit{service worker}, após registrar o \textit{service worker} o navegador inicia a etapa de instalação em segundo plano ??????????.

Na etapa de instalação, é onde se normalmente é armazenado recursos estáticos em cache. Se todos os recursos forem salvos em cache corretamente, o service worker estará instalado. Se no momento de guardar os recursos, ocorrer alguma falha, a etapa de instalação não será finalizada corretamente, e portanto, o service worker não será ativado. Finalizando a etapa de instalação, é iniciada a fase de ativação, é nesse evento onde se gerencia o cache da aplicação e deleta coisas antigas de versões anteriores \cite{servicework} ????????.

Após a etapa de ativação, o service worker gerenciará as páginas dentro do seu escopo, com exceção da página que registrou o service worker, onde ela só será controlada se for carregada novamente. Enquanto o service worker estiver controlando as páginas, ele poderá assumir dois estados: tratando de eventos de busca e mensagens que as páginas possam gerar, ou encerrado, para economizar memoria do dispositivo \cite{servicework}.


Abaixo 	a \autoref{f_c4_sw_ciclo} mostra uma versão minimalista do ciclo de vida do service worker, em sua primeira instalação.
\newpage
\begin{figure}[!htpb]
	\centering
	\caption{Ciclo de Vida do Service Worker}
	\includegraphics[width=7cm]{images/sw-lifecycle.png}\\
	Fonte:?????????????
 	\label{f_c4_sw_ciclo}
\end{figure}

\subsection{Rápido}
A maioria dos dados das \ac{PWA}s é salvo no armazenamento do dispositivo no primeiro acesso. A próxima vez que o usuário acessá-la, a aplicação fará o download de poucos dados. Este é um recurso útil para pessoas que possuem conexão com a internet limitada. O aplicativo é mais confiável do que apenas um site e permite que você envie notificações para seus usuários, mesmo depois que o aplicativo for fechado. Uma vez armazenado em um dispositivo, leva muito menos tempo para ser reativo do que um site comum que precisa buscar e carregar tudo de novo ???????????????.

\subsection{Integração e Engajamento}

As \ac{PWA}s são instaláveis e podem ser exibidos na tela inicial do usuário, sem a necessidade de uma loja de aplicativos, como Google Play Store ou Apple Store. Elas oferecem uma experiência imersiva em tela cheia com a ajuda do arquivo de manifesto do aplicativo web.

O Manifesto do Aplicativo Web, é um arquivo no formato \ac{JSON}, que permite controlar como a aplicação irá aparecer e como ela será iniciada, ela conterá também informações relevantes a aplicação, assim fazendo com que o navegador entenda que a aplicação é uma \ac{PWA} e assim o navegador apresentará uma mensagem para o usuário para que se possa instalar o app na tela inicial do \textit{smartphone} \cite{manifest}.

\begin{figure}[!htpb]
	\centering
	\caption{Adicionar a tela inicial}
	\includegraphics[width=12cm]{images/add-to-home-screen.png}\\
    Fonte:????????????????
 	\label{f_c4_add_home}
\end{figure}

Existem algumas configurações disponíveis no arquivo de manifesto, sendo algumas bem importantes como:

\begin{itemize}
	\item background color - Define a cor de fundo esperada para a aplicação. Esse valor repete o que já está disponível no CSS do site, mas pode ser usado pelos navegadores para desenhar a cor de fundo de um atalho quando o manifesto está disponível antes de a folha de estilo ser carregada. Isso cria uma transição suave entre o carregamento da aplicação e o carregamento do conteúdo do site \cite{manifestfile}.
	\item description - Fornece uma descrição geral do que a aplicação representa
	\item lang - Especifica o idioma principal para as propriedades, "name" e "short name"
	\item dir - Especifica a direção do texto principal para os membros "name", "short name" e "description". Juntamente com o membro "lang", ajuda na exibição correta dos idiomas da direita para a esquerda \cite{manifestfile}.
	\item display - Define o modo de exibição da aplicação.
	\item icons - Especifica uma lista de arquivos de imagem que podem servir como ícones de aplicativo, configurando de acordo com a resolução de tela do dispositivo.
	\item related applications - Define uma lista de aplicativos nativos que podem ser instalados ou acessíveis via plataforma externa, como Google Play Store, esses aplicativos podem ser versões alternativas da aplicação \ac{PWA} \cite{manifestfile}.
	\item short name - Fornece um nome curto legível para o aplicativo. Isso é feito quando não há espaço suficiente para exibir o nome completo da aplicação, como as telas iniciais dos dispositivos.
	\item theme color - Define a cor do tema padrão para a aplicação. Isso às vezes afeta o modo como o sistema operacional o exibe.
\end{itemize}

\section{Vantagens}
\subsection{Fácil Instalação e Atualização}
Como já foi apresentado, para adicionar ou instalar uma aplicação \ac{PWA}, os usuários simplesmente precisam abrir seu site progressivo em um navegador em seu dispositivo móvel. Em algum momento, eles serão solicitados a adicionar o aplicativo em sua tela inicial, ou os usuários podem adicionar o próprio \ac{PWA} no menu do navegador. Bem simples em comparação a aplicativos nativos, em que o usuário necessita ir a uma plataforma de aplicativos e instalar \cite{pwabenefits}.

Quanto às atualizações, os usuários de aplicações \ac{PWA} não precisam atualizar seu aplicativo toda vez que for lançada uma nova versão, eles sempre terão o mais novo.

\subsection{Custos Reduzidos de Desenvolvimento e Suporte}
Não precisa criar uma solução diferente para cada plataforma, pois o mesmo \ac{PWA} funciona no \textit{Android} e no \textit{iOS} e cabe em qualquer dispositivo. Devido à atualização fácil e simultânea, existirá apenas uma versão do aplicativo circulando. O que significa que não se gastará custos extras suportando várias versões \cite{pwabenefits}.

\subsection{Rápido, Leve e Seguro}
Com o \ac{PWA} implementado, a aplicação pode ser carregada em poucos segundos, o que é possível graças aos dados armazenados em cache pelos. Sendo menores em tamanho do que os aplicativos móveis nativos, as \ac{PWA} são mais leves e eficientes e usam menos capacidade e dados do dispositivo. Ao mesmo tempo, eles fornecem uma experiência de usuário próxima à de um aplicativo nativo \textit{Service Workers}. Além disso todas as aplicações \ac{PWA} funcionam via HTTPS, o que significa segurança extra e nenhum acesso não autorizado aos dados \cite{pwa2} ??????????????.

\subsection{Não necessita de nenhuma loja de aplicativos}
Para publicar uma aplicativo nativo para dispositivo móvel é necessário enviá-lo para alguma loja de aplicativos, como, Google Play Store ou Apple Store, e que ainda vai passar por um processo de análise do aplicativo. Com uma \ac{PWA}, não há necessidade de aguardar o término do período de moderação e como já mencionado, para instalar a aplicação progressiva, os usuários só precisarão abrir o website e clicar em "Adicionar à tela inicial". Embora se comportando como um aplicativo nativo, a \ac{PWA} ainda seria uma página da Web, clicável e compartilhável, e também é indexada pelo Google \cite{pwabenefits}.

\subsection{Customizável}
Se o usuário estiver sem conexão e quiser acessar novas páginas que não foram guardadas em cache, a aplicação não irá travar, mas mostrará uma mensagem personalizada, como na \autoref{f_c4_pwa_custom}:

\newpage
\begin{figure}[!htpb]
	\centering
	\caption{Telas Customizadas}
	\includegraphics[width=13cm]{images/pwa_custom.png}\\
	Fonte:??????????
 	\label{f_c4_pwa_custom}
\end{figure}

\section{Desvantagens}
\subsection{Funcionalidades Limitadas}
As \ac{PWA}s ainda são sites e ainda não suportam todas as funcionalidades que os aplicativos nativos podem oferecer. Elas têm acesso limitado a recursos dos dispositivos. Além disso, as \ac{PWA}s podem oferecer um nível de notificação menos personalizado se comparado aos aplicativos nativos.

\subsection{Limitações para o iOS}
No momento, ainda há uma lacuna entre as \ac{PWA}s para Android e iOS. Embora os PWAs estejam disponíveis para usuários do iOS, nem todos os recursos que funcionam em dispositivos Android são oferecidos para iOS, no entanto, existe um movimento na equipe por trás do iOS que aos poucos estão implementando recursos para serem usados via \ac{PWA}, e provavelmente os usuários do iOS irão receber as mesmas funcionalidades das aplicações progressivas disponíveis no Android \cite{pwaios}.%Introdução ao PWA
\chapter{Análise comparativa entre PWAs e Aplicações Móveis Nativas}

As \ac{PWA}s provaram ser muito úteis, e foram apresentados diversos recursos disponíveis em dispositivos móveis, em que as \ac{PWA}s dão suporte. No entanto, eles não estão aqui para tomar o lugar dos aplicativos nativos, mas para corrigir alguns problemas, como a compatibilidade entre plataformas e funcionamento em baixa conectividade e até mesmo offline.

Tendo isso em vista, esse capítulo irá abordar, o que são as aplicações para dispositivos móveis nativas, e uma comparação levando em consideração as vantagens e desvantagens entre desenvolver uma aplicação \ac{PWA} e um aplicativo nativo.

\section{Aplicação Nativa}
Os aplicativos nativos são desenvolvidos para um dispositivo específico ou uma plataforma, como Android e iOS, e podem interagir e aproveitar os recursos fornecidos por esse dispositivo específico. Esses aplicativos oferecem acesso mais rápido a diferentes recursos do dispositivo, como câmera, microfone, agenda, localização e etc.
Para usa-lós é necessário realizar a instalação do aplicativo no dispositivo desejado, normalmente os aplicativos se encontram nas lojas das plataformas, mas também é possível instalar via pacotes externos \cite{native}.

Ao desenvolver uma aplicação nativa, é necessário muitas vezes usar linguagens de programação específicas para a plataforma, elas permitem ao desenvolvedor ter acesso às funcionalidades de \ac{API} e hardware do dispositivo. As linguagens de programação disponíveis para desenvolver aplicativos nativos, varia para cada plataforma \cite{native}. Abaixo alguns exemplos das linguagens de programação disponíveis:

\begin{itemize}
	\item Java - Linguagem oficial do sistema operacional Android, usado para criar aplicativos nativos para a plataforma.
	\item Kotlin - Linguagem mais recente e semelhante ao Java, também para criar aplicativos nativos para Android.
	\item Objective-C - Linguagem principal para criar \textit{software} para dispositivos iOS.
	\item Swift -  Linguagem lançada da Apple para a criação de \textit{software} para o iOS, propõe-se a ser mais simples de implementar do que o Objective-C.
\end{itemize}

\section{Comparativo PWA e Aplicação nativa}
\subsection{Criação da aplicação e Distribuição}
\subsection*{Aplicação Nativa}
O desenvolvimento de uma aplicação móvel nativa para Android e iOS requer duas equipes, uma para cada sistema. Mesmo que os aplicativos de ambos os sistemas sejam desenvolvidos ao mesmo tempo, ainda demorará mais para garantir que a funcionalidade seja a mesma para os dois aplicativos. Tudo isso significa tempo e custos consideráveis necessários para criar um aplicativo.

O envio e aprovação via App Stores é uma parte separada da distribuição do aplicativo móvel nativo. O produto terá que passar por um período de moderação, pela equipe da plataforma, o que geralmente leva algum tempo. Para a Google Play Store, pode levar algumas horas, enquanto na Apple App Store pode levar de dois a quatro dias. Esse tempo de moderação, acaba impactando no atraso da distribuição da aplicação.

\subsection*{PWA}
A criação de uma \ac{PWA} requer apenas uma equipe de desenvolvimento da Web, pois na verdade é um site, embora com alguns recursos nativos dos dispositivos móveis. A validação pelas lojas não é necessária, pois se está criando um \textit{website}. Não é necessário enviar a aplicação para nenhuma loja nem esperar que ele seja aprovado. Depois que a \ac{PWA} é construída e publicada na Web, ela está pronto para ser usada.

\subsection{Instalação}
\subsection*{Aplicação Nativa}
Basicamente o fluxo de instalação é, ir em uma loja de aplicativos, procurar a aplicação e realizar o download, e após isso, a instalação no dispositivo móvel, e por fim o aplicativo estará disponível para uso.

\subsection*{PWA}
Com a \ac{PWA} é necessário abrir um navegador, acessar o site da aplicação, irá ser apresentado uma mensagem, perguntando se quer adicionar a aplicação para a tela inicial do dispositivo, confirmando, a aplicação irá ser instalada.

\subsection{Engajamento do Usuário}
Uma das ferramentas de engajamento mais poderosas é a \textit{Push Notifications}. São mensagens entregues por meio de um aplicativo instalado nos dispositivos, nos dispositivos móveis ou nos desktops dos usuários, para alertar seus usuários sobre novas chegadas de ações, vendas ou outras notícias.

\subsection*{Aplicação Nativa}
Em aplicativos móveis nativos, a disponibilidade do recurso de notificações por \textit{push} não depende do sistema operacional ou do modelo do dispositivo. Os usuários os receberão independentemente desses fatores.

\subsection*{PWA}
Nas \ac{PWA}, as \textit{Push Notifications} também estão disponíveis, mas apenas para o Android. Isso é possível graças aos \textit{service workers} eles podem enviar notificações quando uma aplicação \ac{PWA} não está em execução.

\subsection{Funcionamento Offline}
\subsection*{Aplicação Nativa}
Quando estamos falando sobre o modo offline do aplicativo nativo, assumimos que ele opera da mesma maneira que na conexão. O ponto é que um aplicativo nativo mostra o conteúdo e a funcionalidade que ele conseguiu armazenar em \textit{cache} quando a conexão ainda estava lá. Isso está disponível devido ao armazenamento local e à sincronização suave de dados com a nuvem.

\subsection*{PWA}
Nas aplicações \ac{PWA}, os usuários também podem aproveitar o modo offline. Quando ativadas, as páginas mostram o conteúdo pré-carregado ou carregado, que é fornecido com os \textit{service workers}. No entanto, o modo offline nas \ac{PWA}s é um pouco mais lento em comparação a um aplicativo móvel nativo, pois ele é implementado de forma diferente. Ao mesmo tempo, a diferença entre os dois tipos de aplicativos não é tão drástica.%Estudo Comparativo (Tabela evidenciando a diferença MOBILE x WEB x PWA)
\chapter{\textbf{Estudo de Caso}}
Este capítulo visa demonstrar como desenvolver uma aplicação com suporte à \ac{PWA}. Basicamente o app consiste em 3 telas. Será apresentada uma aplicação que utiliza alguns recursos de uma \ac{PWA}, como, funcionamento \textit{offline}, câmera do dispositivo, localização atual, notificações e \textit{background sync}. Para isso foi utilizado algumas ferramentas como \textit{Firebase}, \textit{Workbox} e \textit{IndexedDB}.

\section{Ferramentas Utilizadas}
\subsection{Firebase}
De acordo com \cite{firebase} "O Firebase é uma plataforma do Google que contém várias ferramentas e uma excelente infraestrutura para ajudar desenvolvedores web e mobile a criar aplicações de alta qualidade e performance". Para o desenvolvimento do app, serão utilizados alguns recursos disponibilizados pelo firebase:

\begin{itemize}
	\item \textbf{Realtime Database}: Banco de dados \textit{NoSQL} hospedado em nuvem.
	\item \textbf{Storage}: Banco de dados utilizado para guardar arquivos de mídia, como imagens, vídeos e áudio.
	\item \textbf{Notifications}: Serviço de envio de notificações para usuários conectados.
\end{itemize}

\subsection{IndexedDB}
O \textit{indexedDB} é um banco de dados disponibilizado pelo \textit{browser}, segundo \cite{indexdb} "\textit{IndexedDB} é uma \ac{API} para armazenamento \textit{client-side} de quantidades significantes de informações e buscas com alta performance por índices". \textit{IndexedDB} é a solução para grande porção de dados estruturados.

\subsection{Workbox}
O \textit{workbox} é uma ferramenta, que vai nos ajudar a manipular, de forma mais simples, o \textit{service worker} da aplicação. Segundo \cite{workbox} "O \textit{workbox} é um conjunto de bibliotecas e módulos que facilitam o armazenamento em \textit{cache} e aproveitam ao máximo os recursos usados para criar uma \ac{PWA}".

\section{Conhecendo o Aplicativo}
Para esse projeto foi desenvolvido um aplicativo que realiza postagens. Cada uma dessas postagem deve possuir além da  imagem, uma descrição e a localização de onde foi realizado tal evento. O aplicativo desenvolvido têm três telas:
\begin{itemize}
	\item \textbf{Tela Inicial}: É  apresentação do aplicativo, ela reúne todas as postagens feitas pelos os usuários, e um botão de ação, para criar uma nova postagem no aplicativo. Como é visto na 
	\autoref{telaInicial}.
	\begin{figure}[!htpb]
		\centering
		\caption{Tela Inicial}
		\frame{\includegraphics[scale=0.20]{images/tela_inicial.png}}\\
		{\footnotesize Fonte: (Elaborado Pelo Autor, 2019)}
		\label{telaInicial}
	\end{figure}

\newpage
	\item \textbf{Tela de Criação de Postagens}: A \autoref{telaDePost} mostra como é a tela que possibilita ao usuário criar uma postagem.
	\begin{figure}[!htpb]
		\centering
		\caption{Tela de Criação de Post}
		\frame{\includegraphics[scale=0.07]{images/tela_post.png}}\\
		{\footnotesize Fonte: (Elaborado Pelo Autor, 2019)}
		\label{telaDePost}
	\end{figure}
	
	\item \textbf{Tela de Ajuda}: É uma página para auxiliar o usuário na navegação do aplicativo. Ainda existem botões de ação que simulariam o entrar em contato com a equipe de SAC, como mostrado na \autoref{telaDeAjuda}.
	\begin{figure}[!htpb]
		\centering
		\caption{Tela de Ajuda}
		\frame{\includegraphics[scale=0.14]{images/tela_ajuda.png}}\\
		{\footnotesize Fonte: (Elaborado Pelo Autor, 2019)}
		\label{telaDeAjuda}
	\end{figure}
\end{itemize}

\section{Configurações Inicias Do Aplicativo com PWA}

\subsection*{Adicionando suporte à PWA}
 Para o aplicativo desenvolvido usar recursos PWA, tem-se que adicionar um  arquivo de configuração do \textit{Service Worker}.

\subsection{Arquivo de Manifesto}
O manifesto da nossa aplicação é configurado por meio de um arquivo de texto, onde possibilita adicionar informações relevantes sobre o aplicativo. Com o arquivo de manifesto configurado nossa aplicação agora pode ser adicionada a tela inicial \cite{manifestfile}. Abaixo a \autoref{f_c4_pwa_custom} que ilustra isso:

\begin{figure}[!htpb]
	\centering
	\caption{Adicionando App}
	\frame{\includegraphics[scale=0.2]{images/pwa_add.jpeg}}\\
	{\footnotesize Fonte: (Elaborado Pelo Autor, 2019)}
	\label{f_c4_pwa_custom}
\end{figure}

E para exemplificar, no \autoref{lst:arquivoManifest} abaixo é mostrado como o manifesto da aplicação ficou configurado:

\newpage

\begin{lstlisting}[frame=single,label=lst:arquivoManifest,caption=Arquivo Manifesto, basicstyle=\footnotesize,]
{
  "name": "TCC App - Progressive Web App",
   "short_name": "PWA TCC",
   "icons": [
 {
   "src": "/src/images/icons/app-icon-48x48.png",
   "type": "image/png",
   "sizes": "48x48"
  },
...
],
"start_url": "/index.html",
"scope": ".",
"display": "standalone",
"orientation": "portrait-primary",
"background_color": "#fff",
"theme_color": "#3c3f50",
"description": "A app, implementing a lot of PWA features.",
"dir": "ltr",
"lang": "en-US"
}
\end{lstlisting}
 \vspace{-0.75cm}
\begin{center}
	\captionof*{lstlisting}{Fonte: (Elaborado Pelo Autor, 2019)}
\end{center}

\subsection{Service Worker}
Como o \textit{service worker} é um \textit{script} do navegador que executa em segundo plano. No primeiro momento vamos apenas configura-lo para fazer cache de alguns recursos úteis a nossa aplicação, como, arquivos \textit{javascript}, css, imagens e páginas html do aplicativo. Para configurar o \textit{service worker}, será criado um arquivo \textit{javascript}, que utilizará uma biblioteca externa chamada \textit{workbox}. O \textit{Workbox} nós ajudará a configurar o \textit{service worker} de maneira mais fácil, já que ele abstrai, muito código \cite{workbox}. O \autoref{lst:serviceWoker} mostra a configuração inicial do  \textit{service worker}.

\newpage

\begin{lstlisting}[frame=single,label=lst:serviceWoker,caption=Service Worker, basicstyle=\footnotesize]
workboxSW.router.registerRoute(:googleapis|gstatic)\.com.
,workboxSW.strategies.staleWhileRevalidate({
  cacheName: 'google-fonts',
  cacheExpiration: {
   maxEntries: 3,
   maxAgeSeconds: 60 * 60 * 24 * 30
  }
}));

workboxSW.router.registerRoute('https://cdnjs.cloudflare.com/ajax/libs/material-design-lite/1.3.0/material.indigo-pink.min.css', workboxSW.strategies.staleWhileRevalidate({
 cacheName: 'material-css'
}));

workboxSW.precache([{
    "url": "favicon.ico",
    "revision": "2cab47d9e04d664d93c8d91aec59e812"
  },
  {
    "url": "index.html",
    "revision": "df4415e7f962b5b6025142af8109f6d2"
  },
  {
    "url": "manifest.json",
    "revision": "cbd4ba1a23f99eb419b91b68d7591d3c"
  },
  {
    "url": "offline.html",
    "revision": "89c91dd62199aa07afc6b6d1ca0db9d1"
  },
  ...
]);

\end{lstlisting}

\vspace{-0.75cm}
\begin{center}
\captionof*{lstlisting}{Fonte: (Elaborado Pelo Autor, 2019)}
\end{center}

\section{Recursos Avançados do Service Worker}
\subsection{Cache das Postagens}
Com o \textit{service worker} configurado, um recurso importante do aplicativo é a listagem de postagens, é possível configurar o \textit{service worker}, para fazer o \textit{cache} das postagens a medida que forem sendo carregadas, assim tornando o aplicativo funcional mesmo em modo \textit{offline}. No arquivo de configuração do \textit{service worker}, criaremos um interceptador para a rota de listagem de Postagens e faremos o tratamento para guardar os dados em cache quando necessário. O \autoref{lst:cacheResponse} mostra o código necessário para que isso seja possível:
\begin{lstlisting}[frame=single,label=lst:cacheResponse,caption=Armazendo Post, basicstyle=\footnotesize]
workboxSW.router.registerRoute(`https://${FIREBASE_URL}/posts.json`, function(args) {
return fetch(args.event.request)
.then(function (res) {
var clonedRes = res.clone();
clearAllData('posts')
.then(function () {
return clonedRes.json();
})
.then(function (data) {
for (var key in data) {
writeData('posts', data[key])
}
});
return res;
});
});
\end{lstlisting}

\vspace{-0.75cm}
\begin{center}
	\captionof*{lstlisting}{Fonte: (Elaborado Pelo Autor, 2019)}
\end{center}

Na configuração acima, os dados retornados da requisição, são guardados no \textit{IndexedDB}, na tabela 'posts'. Quando a requisição é retornada, os dados da tabela 'posts' são limpos, para então ser salvos, afim de manter-mos sempre dados validos no \textit{IndexedDB}.

\subsection{Ações Customizadas}
O \textit{service worker} nos permite customizar ações de tratamento, quando nossa aplicação estiver \textit{offline}, e um recurso interessante que será adicionado, é a exibição de uma página customizada, quando o usuário tenta acessar um recurso offline e que ainda não foi cacheado pelo \textit{service worker}, também conhecida como \textit{fallback page} como mostra a \autoref{offline}.

\begin{figure}[!htpb]
	\centering
	\caption{Tela Offline}
	\frame{\includegraphics[scale=0.24]{images/pwa_offline.png}}\\
	{\footnotesize Fonte: (Elaborado Pelo Autor, 2019)}
	\label{offline}
\end{figure}

O \autoref{lst:customizado} mostra configuração do \textit{service worker}, para exibir uma página customizada:

\newpage

\begin{lstlisting}[frame=single,label=lst:customizado,caption=Página offline customizada, basicstyle=\footnotesize]
workboxSW.router.registerRoute(function (routeData) {
return (routeData.event.request.headers.get('accept').includes('text/html'));
}, function(args) {
return caches.match(args.event.request)
.then(function (response) {
if (response) {
return response;
} else {
return fetch(args.event.request)
.then(function (res) {
return caches.open('dynamic')
.then(function (cache) {
cache.put(args.event.request.url, res.clone());
return res;
})
})
.catch(function (err) {
return caches.match('/offline.html')
.then(function (res) {
return res;
});
});
}
})
});
\end{lstlisting}

\vspace{-0.75cm}
\begin{center}
	\captionof*{lstlisting}{Fonte: (Elaborado Pelo Autor, 2019)}
\end{center}

\subsection{Sincronização de Dados em Background}
Um recurso interessante que o aplicativo possuí é o envio de postagens, mesmo o usuário estando \textit{offline}, o \textit{service worker}, possui um evento de \textit{sync}, onde nele é possível verificar se o aplicativo está com acesso a internet e realizar uma determinada ação \cite{servicework}. 
\newline{}
Ao cadastrar uma postagem, será salvo no \textit{IndexedDB}, o body da requisição para o servidor na tabela 'sync-posts', após isso será registrado um evento 'sync', com a tag 'sync-new-posts' no \textit{service worker}, para que ele trate-o logo mais. O \autoref{lst:sync} com o código das configurações:
\begin{lstlisting}[frame=single,label=lst:sync,caption=Sicronização Em Background, basicstyle=\footnotesize]
if ('serviceWorker' in navigator && 'SyncManager' in window) {
navigator.serviceWorker.ready
.then(function (sw) {
var post = {
id: new Date().toISOString(),
title: titleInput.value,
location: locationInput.value,
picture: picture,
rawLocation: fetchedLocation
};
writeData('sync-posts', post)
.then(function () {
return sw.sync.register('sync-new-posts');
})
.then(function () {
var snackbarContainer = document.querySelector('#confirmation-toast');
var data = {message: 'Seus post foi salvo para a sincronizaacaoo!'};
snackbarContainer.MaterialSnackbar.showSnackbar(data);
})
.catch(function (err) {
console.log(err);
});
});
} else {
sendData();
}
\end{lstlisting}
\vspace{-0.75cm}
\begin{center}
	\captionof*{lstlisting}{Fonte: (Elaborado Pelo Autor, 2019)}
\end{center}


Para o \textit{service worker} tratar o evento que acabou de ser transmitido, é necessário implementar uma função. Para isso verificaremos se o evento transmitido terá a tag 'sync-new-posts', que foi passada previamente, e assim podemos tratar os dados da forma que quisermos, nesse caso, vai ser chamada os dados do \textit{IndexedDB}, referente a tabela 'sync-posts' e enviaremos os dados para o servidor \textit{Firebase}, e por fim, limpamos os dados da tabela 'sync-posts', que acabaram de ser salvos. A seguir o código da função implementada é mostrada no \autoref{lst:limparDados}:

\begin{lstlisting}[frame=single,label=lst:limparDados,caption=Limpando Dados, basicstyle=\footnotesize]
self.addEventListener('sync', function(event) {
console.log('[Service Worker] Background syncing', event);
if (event.tag === 'sync-new-posts') {
console.log('[Service Worker] Syncing new Posts');
event.waitUntil(
readAllData('sync-posts')
.then(function(data) {
for (var dt of data) {
var postData = new FormData();
postData.append('id', dt.id);
postData.append('title', dt.title);
postData.append('location', dt.location);
postData.append('rawLocationLat', dt.rawLocation.lat);
postData.append('rawLocationLng', dt.rawLocation.lng);
postData.append('file', dt.picture, dt.id + '.png');
fetch(`https://${FIREBASE_FUNCTION_URL}/storePostData`, {
method: 'POST',
body: postData
})
.then(function(res) {
console.log('Sent data', res);
if (res.ok) {
res.json()
.then(function(resData) {
deleteItemFromData('sync-posts', resData.id);

\end{lstlisting}
\vspace{-0.75cm}
\begin{center}
	\captionof*{lstlisting}{Fonte: (Elaborado Pelo Autor, 2019)}
\end{center}

\subsection{Notificações}
Notificações é uma ferramenta importante para informar o usuário sobre um determinado evento, tendo isso em vista, o aplicativo desenvolvido vai enviar notificações para os usuários quando for criada uma nova postagem. Para a criação do fluxo de notificação foi usado o \textit{Firebase} e configurações no arquivo do \textit{service worker}. Em um primeiro momento é necessário verificar se o usuário deseja receber notificações, por isso, o aplicativo possui um botão que habilita as notificações. Como mostra a figura \autoref{notificacao}.

\begin{figure}[!htpb]
	\centering
	\caption{Habilitar Notificações}
	\frame{\includegraphics[scale=0.15]{images/pwa_notification.png}}\\
	{\footnotesize Fonte: (Elaborado Pelo Autor, 2019)}
	\label{notificacao}
\end{figure}

Este botão só estará disponível se o browser possuir suporte as notificações, para isso foi adicionado uma verificação, em que esconde o botão caso as notificações não estiverem disponíveis. O \autoref{lst:notificacoes} mostra a lógica para que isso aconteça.

\begin{lstlisting}[frame=single,label=lst:notificacoes,caption=Verificando suporte notificações, basicstyle=\footnotesize]
if ('Notification' in window && 'serviceWorker' in navigator) {
for (var i = 0; i < enableNotificationsButtons.length; i++) {
enableNotificationsButtons[i].style.display = 'inline-block';
enableNotificationsButtons[i].addEventListener('click', askForNotificationPermission);
}
}
\end{lstlisting}
\vspace{-0.75cm}
\begin{center}
	\captionof*{lstlisting}{Fonte: (Elaborado Pelo Autor, 2019)}
\end{center}

\newpage
O \textit{browser} tendo suporte as notificações, é necessário a permissão do usuário para podermos utiliza-las. Como mostra o \autoref{lst:permissao}.

\begin{lstlisting}[frame=single,label=lst:permissao,caption=Pedindo permissão ao usuário, basicstyle=\footnotesize]
function askForNotificationPermission() {
Notification.requestPermission(function(result) {
console.log('User Choice', result);
if (result !== 'granted') {
console.log('No notification permission granted!');
} else {
configurePushSub();
}
});
}
\end{lstlisting}
\vspace{-0.75cm}
\begin{center}
	\captionof*{lstlisting}{Fonte: (Elaborado Pelo Autor, 2019)}
\end{center}

O usuário dando permissão para o aplicativo usar as notificações, podemos configurar-las e criar uma inscrição no \textit{Firebase}, essa inscrição vai auxiliar no envio das notificações, como uma identificação para o dispositivo do usuário e enviar a notificação para eles. Para essa ação, foi criada uma função que verifica se o dispositivo já possui uma inscrição,conforme pode ser visto no Quadro 9.


\begin{lstlisting}[frame=single,label=lst:inscricao,caption=Configurando Push Notifications, basicstyle=\footnotesize]
function configurePushSub() {
if (!('serviceWorker' in navigator)) return;
var reg;
navigator.serviceWorker.ready
.then(function(swreg) {
reg = swreg;
return swreg.pushManager.getSubscription();
})
.then(function(sub) {
if (sub === null) {
// Create a new subscription
var convertedVapidPublicKey = urlBase64ToUint8Array(PUBLIC_KEY);
return reg.pushManager.subscribe({
userVisibleOnly: true,
applicationServerKey: convertedVapidPublicKey
});
}
})
.then(function(newSub) {
return fetch(`https://${FIREBASE_URL}/subscriptions.json`, {
method: 'POST',
headers: {
'Content-Type': 'application/json',
'Accept': 'application/json'
},
body: JSON.stringify(newSub)
})
})
.then(function(res) {
if (res.ok) {
displayConfirmNotification();
}
})
.catch(function(err) {
console.log(err);
});
}
\end{lstlisting}
\vspace{-0.75cm}
\begin{center}
	\captionof*{lstlisting}{Fonte: (Elaborado Pelo Autor, 2019)}
\end{center}


\newpage
No retorno da requisição para o \textit{Firebase}, se tudo ocorrer bem, é chamada uma função que exibe para o usuário uma notificação, pré configurada, conforme exibido no Quadro 10..
\begin{lstlisting}[frame=single,label=lst:pushNotificacao,caption= Notificação Pré configurada, basicstyle=\footnotesize]
function displayConfirmNotification() {
if ('serviceWorker' in navigator) {
var options = {
body: 'You successfully subscribed to our Notification service!',
icon: '/src/images/icons/app-icon-96x96.png',
image: '/src/images/sf-boat.jpg',
dir: 'ltr',
lang: 'en-US', // BCP 47,
vibrate: [100, 50, 200],
badge: '/src/images/icons/app-icon-96x96.png',
tag: 'confirm-notification',
renotify: true,
actions: [
{ action: 'confirm', title: 'Okay', icon: '/src/images/icons/app-icon-96x96.png' },
{ action: 'cancel', title: 'Cancel', icon: '/src/images/icons/app-icon-96x96.png' }
]
};
navigator.serviceWorker.ready
.then(function(swreg) {
swreg.showNotification('Successfully subscribed!', options);
});
}
}
\end{lstlisting}
\vspace{-0.75cm}
\begin{center}
	\captionof*{lstlisting}{Fonte: (Elaborado Pelo Autor, 2019)}
\end{center}

\newpage
A \autoref{configuracaoNotificacao}, mostra  o exemplo, de como será exibido a notificação:

\begin{figure}[!htpb]
	\centering
	\caption{Notificação Pré Configurada}
	\frame{\includegraphics[scale=0.15]{images/pwa_default_notification.jpeg}}\\
	{\footnotesize Fonte: (Elaborado Pelo Autor, 2019)}
	\label{configuracaoNotificacao}
\end{figure}

Por fim, para exibir as notificações, é necessário, configurar o service worker. O \textit{service worker} possui o evento 'push', onde é configurada uma função para o tratamento das notificações, conforme mostra o \autoref{lst:notificacaoPersonalixada}.
\begin{lstlisting}[frame=single,label=lst:notificacaoPersonalixada,caption= Notificação personalizada, basicstyle=\footnotesize]
self.addEventListener('push', function(event) {
console.log('Push Notification received', event);
var data = {title: 'New!', content: 'Something new happened!', openUrl: '/'};
if (event.data) data = JSON.parse(event.data.text());
var options = {
body: data.content,
icon: '/src/images/icons/app-icon-96x96.png',
badge: '/src/images/icons/app-icon-96x96.png',
data: {
url: data.openUrl
event.waitUntil(self.registration.showNotification(data.title, options));
\end{lstlisting}
\vspace{-0.62cm}
\begin{center}
	\captionof*{lstlisting}{Fonte: (Elaborado Pelo Autor, 2019)}
\end{center}%Estudo de Caso
\chapter{\textbf{Conclusão}}

Analisando os resultados apresentados, as \ac{PWA}s, não chegaram para substituir as aplicações nativas, essa não é a sua proposta, mas sim facilitar o desenvolvimento de aplicações web, com recursos nativos de um dispositivo móvel, em que normalmente só era possível com aplicações nativas, e vão além disso, implementando recursos como o funcionamento offline e tratamentos customizados. O uso ou não da \ac{PWA} vai depender exclusivamente do negócio em que a aplicação vai ser direcionada, caso ela vá possuir diversas funcionalidades nativas, o ideal é desenvolver uma aplicação sem o uso da \ac{PWA}, entretanto, se uma \ac{PWA} possuir tais recursos, eu recomendo o uso da \ac{PWA}, por facilitar o desenvolvimento.

Por ser algo muito recente a maioria dos livros que abordam sobre as \ac{PWA}
começaram a serem publicados em meados de 2016, oferecendo na maioria das vezes uma visão
mas geral do assunto e com pouco foco na parte prática, com isso se fez necessário
procurar conteúdo nas mais variáveis fontes, como sites, artigos, e no próprio \textit{GitHub} . Com o intuito
de definir quais seriam os objetivos desse trabalho. Com a pesquisa bibliográfica foi
possível entender melhor a dimensão que as \ac{PWA} podem ter no futuro.

É importante salientar que entre os objetivos definidos nesse trabalho, a configuração do ambiente de
desenvolvimento foi o mais complexo. E a causa disso, foi porque a maioria do material disponível na Internet estava incompleta, e não mostrava como configurar por completo o
ambiente, outra ocorrência comum era o aparecimento de erros no ambiente durante
a execução de algum comando nos \textit{service workers}.


Sem dúvida as \ac{PWA}s, irão melhorar o engajamento do usuário, por exemplo, quando se utiliza um app mobile nativo, tem todo um passo à passo para começar a usá-lo. Primeiro tem que ir na loja de aplicativos do dispositivo móvel, fazer o download do app esperar a conclusão da instalação e após isso poder usá-lo, já com uma \ac{PWA} o usuário basta abrir um navegador, de sua preferência, e acessar a url do app, e pronto, já está disponível para utilização. Além do que, não é necessário 
esperar dias para que o novo app esteja disponível na loja de app, por exemplo, na Google Play.


% Conclusao
% ----------------------------------------------------------
% PARTE I
% ----------------------------------------------------------
%\part{Preparação da pesquisa}

% Capitulo com exemplos de comandos inseridos de arquivo externo 
%\include{fundamentacao-teorica}
%\include{trabalho-relacionado}
%\include{abntex2-modelo-include-comandos}
%\include{objetivos}
% ----------------------------------------------------------
% PARTE II
% ----------------------------------------------------------
%\part{Referenciais Teóricos}


% Capitulo de revisão de literatura
%\chapter{Lorem ipsum dolor sit amet}
%\include{procedimento-metodologico}


% ----------------------------------------------------------
% PARTE III
% ----------------------------------------------------------
%\part{Resultados}
%\include{desenvolvimento-resultados}

%\include{discussao}

%\include{consideracoes-finais}
% ---
% primeiro capitulo de Resultados
%\chapter{Lectus lobortis condimentum}

%\section{Vestibulum ante ipsum primis in faucibus orci luctus et ultrices posuere cubilia Curae}

%\lipsum[21-22]

% ---
% segundo capitulo de Resultados
%\chapter{Nam sed tellus sit amet lectus urna ullamcorper tristique interdum elementum}

%\section{Pellentesque sit amet pede ac sem eleifend consectetuer}

%\lipsum[24]

% ----------------------------------------------------------
% Finaliza a parte no bookmark do PDF para que se inicie o bookmark na raiz e adiciona espaço de parte no Sumário
% ----------------------------------------------------------
\phantompart

% ---
% Conclusão (outro exemplo de capítulo sem numeração e presente no sumário)
%\chapter*[Conclusão]{Conclusão}
%\addcontentsline{toc}{chapter}{Conclusão}

%\lipsum[31-33]

% ----------------------------------------------------------
% ELEMENTOS PÓS-TEXTUAIS
% ----------------------------------------------------------
\postextual

% Referências bibliográficas
\bibliography{bibtex/referencias}

% Glossário (Consulte o manual da classe abntex2 para orientações sobre o glossário)
%\glossary

% Apêndices
%\include{editaveis/apendices}

% Anexos
%\include{editaveis/anexos}

%---------------------------------------------------------------------
% INDICE REMISSIVO
%---------------------------------------------------------------------
\phantompart
\printindex
%---------------------------------------------------------------------
\end{document}
